 \section{Electromagnetic Calorimeter}
Directly outside of the tracking system lies the electromagnetic calorimeter 
(ECAL) of CMS. The driving criteria of the ECAL design is to provide capability
to detect and measure the decay to two photons of the Higgs Boson. The 
ECAL is designed with the objective of a
fast response time, a fine granularity and resistance to the effects of radiation.
Therefore, recently advanced lead tungstate (PbWO$_{4}$) technology was chosen. 
A preshower detector is placed in front of the endcap. Avalanche photodiodes (APDs)
are used as photodetectors in the barrel and vacuum phototriodes (VPTs) in the
endcaps.
The layout of the CMS ECAL is shown in Figure %Include ECAL figure
The barrel part of the ECAL (EB) covers the pseudorapidity range $|\eta|<1.479$
the endcap part covers the rapidity range $1.479<|\eta|<3.0$.
\subsection{Lead Tungstate Crystals}
The ECAL is composed of 75,848 PbWO$_{4}$ crystals: 61,200 mounted in the 
central barrel part, closed by 7,324 crystals in each of the two endcaps. They 
have a high density of 8.28$g/cm^{3}$ and a short radiation length of 0.89cm. 
The Moli\grave{e}re radius (radius of a cylinder containing 90\% of the shower's energy deposit) 
is only 2.2cm. These characteristics result in a fine granularity and a compact
calorimeter. Futhermore, the scintillation decay time of these crystals is of the
same order of magnitude as the LHC bunch crossing time whereby 80\% of 
the light is emitted in 25ns. 
\subsection{Energy Resolution}
The energy resolution in the ECAL can be parameterized as in the following equation:
\begin{displaymath}
(\frac{\sigma}{E}^{2})=(\frac{S}{\sqrt{E}})^{2})+(\frac{N}{E}^{2})+C^{2}
\end{displaymath}
where $S$ is the stochastic term, $N$ the noise term, and $C$ the constant term. 
The individual contributions are described in the following paragraphs.

