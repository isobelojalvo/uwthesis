\section{A Historical Approach to High Energy Physics}
 %In___ Aristotle asked himself what was the everything made u
 The modern understanding of particle physics is generally
 agreed to have began in 1897
 when J. J. Thomson fired "cathode rays" into a magnetic field. 
He observed their circular orbit in this magnetic field and measuring
the radius of the orbit, %%fix wording
 Using the equation, %%%%eqn
 developed by already well established electromagnetic theory, Thomson
 was able approximate the mass of the electron. 
 Thomson already knew these electrons were in some way associated
 to the atom; he hypothesized that the electrons were distributed
 evenly with in the atom, much like frogs floating in a pond.
 In 1899 Rutherford tested this hypothesis by firing a beam of 
 $\alpha$-particles at a thin gold sheet. He observed that most of the
 $\alpha$-particles went straight through the sheet, but a few bounced
 off in various directions; this suggested that the $\alpha$-particles were made mostly of 
 space with an indivisible nucleus that would on occasion interact and
 cause the scattering. Rutherford named the nucleus of the lightest
 atom the "proton". Even if at this time physicists had an idea of the nature%fix
 of the most simple of atoms the relationship between the nucleus and the
 electrons was still not well understood. In 1914 Niels Bohr successfully
 developed a model of the Hydrogen atom by approximating the electron
 as a planet circling the Sun, i.e. the nucleus. This equation, %%%insert eqn for bohr hypothesis
Bohr developed, amazingly, was very accurate in determining the 
Hydrogen spectrum. 

Around the same time that Rutherford was measuring the mass of the
electron, Planck found that by quantizing electromagnetic radiation by
$E=h\nu$ he could escape the ultraviolet catastrophe.%%%%fix this what is UV catastrophe?

In 1905 Einstein took Planck's idea of the quantization of the electromagnetic
radiation a step further and claimed that radiation is intrinsically quantized.
%%addequation
This "photoelectric effect" was long rejected by the physics community until
decisively proven by Compton Scatter
 %%add in Marie Curie

%discovery of pion Dirac
%discovery of positron



%%%%%%%%gluon
%At the time of this measurement, electron-positron experiments regularly observed two-jet events. They indicate the annihilation of the two incoming particles and the subsequent creation of a quark and an antiquark, both of which result in particle jets. If the energy of the collision is sufficiently large, either the quark or the antiquark can emit excess energy in form of a gluon, which also produces a particle jet in the same plane as the two other jets. Only at high energies, the gluon jet appears as a distinct third jet.
%Physicists Sau Lan Wu and Georg �Haimo� Zobernig developed and programmed a method to search for such planar three-jet events among the TASSO data. At low collision energies, their searches produced no results. But when DESY�s PETRA accelerator began to produce collisions at 27.4 GeV, they succeeded. A week later, Bj�rn Wiik presented this first event on behalf of the TASSO collaboration at a physics conference in Bergen, Norway. Shortly thereafter, on June 26, 1979, Wu and Zobernig distributed the above figure in their internal TASSO Note No. 84.