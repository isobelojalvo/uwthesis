\section{Introduction}
 Around 450 B.C. the greek philosopher Democritus created the term 'atom'
 to refer to an indivisible quantum of which all matter is composed.
 Democritus' use of this word was purely philosophical,
 the true scientific study of the building blocks of matter began in the late 19th
 century. In the next section, a selected historical approach which highlights
 important advances in particle physics pertinent to the development of the standard model of physics is taken 
culminating in the recent discovery of a higgs boson.
 \section{A Historical Approach to High Energy Physics}
 The first leap forward in understanding of the
 particle physics field came in 1897 when J. J. Thomson fired what was then known as 'cathode rays' into a magnetic field. 
He observed their circular orbit and by measuring
the radius of the orbit, Thomson
 was able to make the first measurement of the electron mass.
 Thomson already knew these electrons were in some way associated
 to the atom and he hypothesized that the electrons were distributed
 evenly within the atom, much like plums in a pudding.
 In 1899 Rutherford tested this hypothesis by firing a beam of 
 $\alpha$ particles at a thin gold sheet. He observed that most of the
 $\alpha$ particles went straight through the sheet, but a few bounced
 off in various directions; this suggested that the $\alpha$ particles were made mostly of 
 space with a very dense nucleus that would on occasion interact and
 cause the scattering. Rutherford named the nucleus of the lightest
 atom the 'proton'. 
 
 Even if at this time physicists had an idea of the nature
 of the most simple of atoms the relationship between the nucleus and the
 electrons was still not well understood. In 1914 Niels Bohr successfully
 developed a model of the Hydrogen atom by approximating it to a planet
 (the electron) circling the Sun (the nucleus). This rudimentary model that 
Bohr developed was very accurate in determining the 
Hydrogen spectrum, however, was ultimately insufficient
in describing atomic physics. In 1932, about thirty years after
the discovery of the proton, 
Chadwick explained the difference between the mass and charge of larger
atoms (for example $\mathrm{He}_{4}^{2}$) with the discovery of the neutron. 

Around the same time that Rutherford was measuring the mass of the
electron, Planck was performing studies on black-body radiation.
A black-body will absorb all radiation incident
upon it and then re-radiate this energy, in classical physics this
radiation can occur at any wavelength; this classical model lead to an 'ultra-violet catastrophe' 
whereby an ideal black body at thermal equilibrium will emit
radiation with infinite power.
Planck found that if electromagnetic radiation emission was quantized 
based upon the frequency, 
$E=h\nu$, then the ultraviolet catastrophe would be avoided. 
%he could escape the ultraviolet catastrophe. %that an ideal black body at thermal equilibrium will emit radiation with infinite power
In 1905 Einstein took Planck's idea of the quantization of electromagnetic
radiation a step further and claimed that radiation emission from an atom is intrinsically quantized,
$E\leq h\nu-W$, where W is the work function which was found to be the
amount of energy needed to transport an electron from its current bound energy level
to an unbound free-state.
This 'photoelectric effect' was long rejected by the physics community until
decisively proven by Compton Scattering experiments which were performed by Compton
in 1923. These results verified that light is both a particle and a wave.

The proposal of Dirac's equation in 1928 marks the ending of the era
of non-relativistic quantum mechanics. Dirac's famous equation combined
high momentum physics described by Einstein's theory of relativity
with the smallest of distances described by quantum mechanics. Although
the positive energy solutions of Dirac's equation appeared to correctly 
describe electrons experimentally it had negative energy solutions, a 
feature which Dirac did not like at the time. The true triumph of the
theory came in 1932 when Anderson discovered the positron. In the
1940's Feynman and Stuckelberg interpreted the positron and negative energy solutions as 
an electron moving backwards in time. This idea of antiparticles is
of central importance to quantum field theory. 

The first significant theory of the strong force was developed by Yukawa
in 1934. Yukawa believed that the proton and the neutron were
attracted to each other by some sort of field with a heavy mediator called
the meson. By 1937 the pion ($\pi$) and the muon ($\mu$) were identified in cosmic rays;
while the $\pi$ was the meson Yukawa had suggested
 it took much longer to understand the mystery of the $\mu$ as well as 
 to develop an understanding of the  the interactions of fundamental particles.

During the study of nuclear $\beta$ decay in the 1930's a problem was
observed. In $\beta$ decays a nucleus transforms into a slightly lighter
nucleus with the emission of an electron; if the neutron is at rest then the
electron and proton must emerge back to back with equal and opposite
momenta. However, when the energy of the electron was recorded over
many experimental iterations it was shown to vary! A solution to this was
proposed by Pauli that a neutral particle (the neutrino) carries off this missing energy; today
this process is known to be $n\rightarrow p^{+}+e^{-}+\bar{\nu}$. 

%By this time it had been seen that in any interaction charge and energy
%must be conserved% mu- to e- + photon never observed mu = 
%lepton number conservation
%
By 1947 particle physics was, for the most part, well
described and understood but late in 1947 while studying cosmic rays 
incident on a thick lead plate Butler and Rochester 
observed $K^{0}\rightarrow \pi^{+}\pi^{-}$ and then in 1949
Brown observed $K^{+}\rightarrow \pi^{+}\pi^{+}\pi^{-}$. This ushered in a 
slew of particle discovery in the 1950's.%%finish here
During this time many issues arose: 
Did the production of $\Lambda$ via $p+pi^{-}\rightarrow K^{0}+\Lambda$
take only $10^{-23}$ seconds but the decay
$\Lambda\rightarrow p+\pi^{+}$ takes $10^{-10}$ seconds? 
Why was there no $p\rightarrow e^{+}+\gamma$? These discoveries began the development of
conservation laws such as baryon number and strangeness.
Then, in 1961 Gell-Mann arranged baryons in geometric patterns
according to their charge, Q, and strangeness, S, and successfully 
predicted the $\Omega^{-}$. He called this the 'Eightfold Way'
 at the time to was considered an periodic table for particle physics
and initiated many future advancements.
%Gell-Mann --eightfold way

%%%%%%%%gluon
%At the time of this measurement, electron-positron experiments regularly observed two-jet events. They indicate the annihilation of the two incoming particles and the subsequent creation of a quark and an antiquark, both of which result in particle jets. If the energy of the collision is sufficiently large, either the quark or the antiquark can emit excess energy in form of a gluon, which also produces a particle jet in the same plane as the two other jets. Only at high energies, the gluon jet appears as a distinct third jet.
%Physicists Sau Lan Wu and Georg �Haimo� Zobernig developed and programmed a method to search for such planar three-jet events among the TASSO data. At low collision energies, their searches produced no results. But when DESY�s PETRA accelerator began to produce collisions at 27.4 GeV, they succeeded. A week later, Bj�rn Wiik presented this first event on behalf of the TASSO collaboration at a physics conference in Bergen, Norway. Shortly thereafter, on June 26, 1979, Wu and Zobernig distributed the above figure in their internal TASSO Note No. 84.

Despite all the best efforts, an individual quark has not been observed, which led to rampant skepticism about the quark model. %Despite many efforts, no one has ever observed a single, individual 
quark. This produced much skepticism about the quark model 
in the 1960's and 70's. To explain the absence of individual quarks 
the idea of 'quark confinement' was introduced, i.e. quarks are always confined within mesons or baryons. 
In the late 1960's physicists at the Stanford Linear Accelerator (SLAC) %%reference SLAC paper FriedmanKendall
performed deep inelastic scattering experiments to study the sub-structure
of the nucleon. By way of firing an electron at a nucleon and measuring the
scattering of the outgoing electron the experimental results hinted that the
nucleon was actually made up of many point-like constituents. These
point-like constituents would eventually be called 'partons'. 
The discovery of these point-like constituents presented their own problem
as per the Pauli exclusion principal two particles cannot exist in the exact
same state. To solve this issue,
W. Greenberg proposed that quarks come in 3 colors, red, green and blue,
(which have anti-red, anti-green, and anti-blue partners) and that each 
quark bound state is actually colorless. 
The skepticism about the quark model remained %GIM Mechanism?
widespread until the discovery of the J/$\psi$ particle by separate groups
at SLAC and MIT. The J/$\psi$ is the $c\bar{c}$ meson; its ground and excited states were
shown to be well described by Quantum Chromodynamics (QCD). %%Write which symmetry it follows?

A very unanticipated third generation of lepton, the $\tau$-lepton, was 
discovered at SLAC. The $\tau$ has a heavy mass (1.8 GeV) and the lifetime of the $\tau$ is much
shorter than that of the $\mu$; the reconstruction of the $\tau$ is also
more difficult due to the fact that it decays both to leptons ($\tau\rightarrow\mu\bar{\nu_{\tau}}\nu_{\mu}$)
and to hadrons ($\tau\rightarrow \pi^{+}\pi{-}\pi^{-}\nu_{\tau}$).
This discovery of a third lepton generation was unprecedented as there 
had previously been discovered only 2 generations of quarks and 2 
generations of leptons. However, in 1977 a $\Upsilon$ 
resonance was observed via its decay to $\mu^{+}\mu^{-}$ at approximately
9.5 GeV. It was later shown that this peak at 9.5 GeV was actually three $b\bar{b}$ resonances,
with the ground state at 9.4GeV.
With a mass of 4.3GeV, the b-quark is much more massive than the c-quark 
(1.2 GeV); the b-quark's long lifetime and high mass means it is of great 
importance in the discovery of new physics.

By 1979 the production of a 2 jet event, $q\bar{q}$. had already been observed in 
electron-positron collisions at PETRA 
via the process $e^{+}e^{-}\rightarrow q\bar{q}$. 
However, the observation of a 3-Jet event at PETRA was extremely exciting as it
was the first evidence of a high energy quark emitting a gluon via
Bremsstrahlung, $e^{+}e^{-}\rightarrow q\bar{q}g$. 

Fermi theorized that the weak interaction in $\beta$ decays occur at a single point.
However, it was quickly seen that at high energies this theory must fail
and there must exist an intermediate vector boson to mediate the
incredibly weak interaction ($10^{13}$ times weaker than the strong interaction).
Electroweak theory was proposed by Glashow, Weinberg and Salam 
and it predicted the mass of the $W^{\pm}$ and $Z$ bosons. In the 1970s 
CERN began construction of a proton-proton collider, the Super Proton Synchrotron (SPS), that would operate
with a center of mass energy of 540 GeV.
The UA1 detector was constructed on this ring a large
collaboration at CERN led by Carlo Rubbia. In January of 1983 Rubbia announced the discovery of the 
$W$, and 5 months later the $Z$ was reported as well. 

This summary of observations and tests of the standard model brings us 
to July 2nd 2012 when physicists separately from the CMS and ATLAS 
experiments at CERN both independently reported observation of a 
Higgs-like boson at approximately 125 GeV. The discovery of this particle confirmed
that spontaneous symmetry breaking via the Higgs Mechanism give 
the vector bosons their masses. In the following sections the
Standard Model (SM) and an extension of this model, the Minimally Supersymmetric
Standard Model (MSSM), is described in more detail. The discovery of a Higgs-like boson
that couples to bosons and fermions is a major milestone in the story
of particle physics. It remains to be seen where the story will go; a search
for super symmetry is presented with no strong evidence yet to
support this model. However, many important questions about the 
universe remain unanswered and so the saga continues. 