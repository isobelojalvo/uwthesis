\section{A Historical Approach to High Energy Physics}
 %In___ Aristotle asked himself what was the everything made u
 The modern understanding of particle physics is generally
 agreed to have began in 1897
 when J. J. Thomson fired "cathode rays" into a magnetic field. 
He observed their circular orbit in this magnetic field and measuring
the radius of the orbit, %%fix wording
 Using the equation, %%%%eqn
 developed by already well established electromagnetic theory, Thomson
 was able approximate the mass of the electron. 
 Thomson already knew these electrons were in some way associated
 to the atom; he hypothesized that the electrons were distributed
 evenly with in the atom, much like frogs floating in a pond.
 In 1899 Rutherford tested this hypothesis by firing a beam of 
 $\alpha$-particles at a thin gold sheet. He observed that most of the
 $\alpha$-particles went straight through the sheet, but a few bounced
 off in various directions; this suggested that the $\alpha$-particles were made mostly of 
 space with an indivisible nucleus that would on occasion interact and
 cause the scattering. Rutherford named the nucleus of the lightest
 atom the "proton". Even if at this time physicists had an idea of the nature%fix
 of the most simple of atoms the relationship between the nucleus and the
 electrons was still not well understood. In 1914 Niels Bohr successfully
 developed a model of the Hydrogen atom by approximating the electron
 as a planet circling the Sun, i.e. the nucleus. This equation, %%%insert eqn for bohr hypothesis
Bohr developed, amazingly, was very accurate in determining the 
Hydrogen spectrum. 

Around the same time that Rutherford was measuring the mass of the
electron, Planck found that by quantizing electromagnetic radiation by
$E=h\nu$ he could escape the ultraviolet catastrophe.%%%%fix this what is UV catastrophe?


In 1905 Einstein took Planck's idea of the quantization of the electromagnetic
radiation a step further and claimed that radiation is intrinsically quantized.
%%addequation
This "photoelectric effect" was long rejected by the physics community until
decisively proven by Compton Scatter
 %%add in Marie Curie
%Dirac discovery of positron
In the 1930's during the study of nuclear $\Beta$ decay a problem was
observed. In $\Beta$ decay a nucleus transforms into a slightly lighter
nucleus with the emission of an electron; if the neutron is at then the
electron and proton must emerge back to back with equal and opposite
momenta. However, when the energy of the electron was recorded over
many experimental iterations it was shown to vary. A solution to this was
proposed by Pauli that a neutral particle carries off this missing energy; today
this process is known to be $n\rightarrow p^{+}+e^{-}+\overbar{\nu}$. 

By this time it had been seen that in any interaction charge and energy
must be conserved% mu- to e- + photon never observed mu = 
%%%%%%%%gluon
%At the time of this measurement, electron-positron experiments regularly observed two-jet events. They indicate the annihilation of the two incoming particles and the subsequent creation of a quark and an antiquark, both of which result in particle jets. If the energy of the collision is sufficiently large, either the quark or the antiquark can emit excess energy in form of a gluon, which also produces a particle jet in the same plane as the two other jets. Only at high energies, the gluon jet appears as a distinct third jet.
%Physicists Sau Lan Wu and Georg �Haimo� Zobernig developed and programmed a method to search for such planar three-jet events among the TASSO data. At low collision energies, their searches produced no results. But when DESY�s PETRA accelerator began to produce collisions at 27.4 GeV, they succeeded. A week later, Bj�rn Wiik presented this first event on behalf of the TASSO collaboration at a physics conference in Bergen, Norway. Shortly thereafter, on June 26, 1979, Wu and Zobernig distributed the above figure in their internal TASSO Note No. 84.

Despite many efforts, no one has ever observed a single, individual 
quark. This produced much skepticism in the 1960's and 70's;
to explain this the idea of "quark confinement" was introduced which
said that quarks are always confined within mesons or baryons. 
%add in something about gluons
In the late 1960's physicists at the Stanford Linear Accelerator (SLAC) %%reference SLAC paper FriedmanKendall
performed deep inelastic scattering experiments to study the sub-structure
of the nucleon. By way of firing an electron at a nucleon and measuring the
scattering of the outgoing electron the experimental results hinted that the
nucleon was actually made up of many point-like constituents. These
point-like constituents would eventually be called "partons". 
W. Greenberg proposed that quarks come in 3 colors, red, green and blue,
(which have anti-red, anti-green, and anti-blue partners) and that each 
quark bound state is actually colorless. 

This widespread skepticism about the quark model remained 
widespread until the discovery of the J/$\psi$ particle by separate groups
at SLAC and MIT. This bound $c\bar{c}$ ground and excited states were
shown to be well described by Quantum Chromodynamics (QCD). %%Write which symmetry it follows?

A very unanticipated third generation of lepton, the $\tau$-lepton, was 
discovered at SLAC. Due to its heavy mass (1784 MeV) the lifetime of the $\tau$ is much
shorter than that of the $\mu$; the reconstruction of the $\tau$ is also
more difficult due to it decay both to leptons ($\tau\rightarrow\mu\bar{\nu_{\tau}}\nu_{\mu}$)
and to hadrons $$.%%%%quote tau decay
This discovery of a third lepton generation was unprecedented as there 
had previously been discovered only 2 generations of quarks. However,
in 1977 a %$$\psi???
resonance was observed via its decay to $\mu^{+}\mu^{-}$ at approximately
9.5 GeV. It was later shown that this peak at 9.5 GeV was actually three $b\bar{b}$ resonances.
The b-quark is much more massive than the c-quark; the b-quarks decay and
its huge importance in the discovery of new physics is outlined in section %%%%. 

The observation of a 3-Jet event in 1979 was extremely exciting as it may
%%add in gluons

This summary of observations and hints of the Standard Model confirmation brings us 
to July 2nd 2012 when physicists separately at the CMS and ATLAS 
experiments at CERN both independently reported observation of a 
Higgs-like boson at approximately 125 GeV. In the following sections the
Standard Model (SM) and an extension of this model, the Minimally Supersymmetric
Standard Model (MSSM), is described in more detail. The discovery of a Higgs-like boson
that couples to bosons and fermions is a major milestone in the story
of particle physics. It remains to be seen where the story will go; a search
for the super symmetry is presented with no strong evidence yet to
support this model. However, many important questions about the 
universe remain unanswered and so the saga continues. 