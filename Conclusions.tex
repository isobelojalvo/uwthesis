\chapter{Conclusions}
In summary, this thesis has presented two results:
a standard model cross section measurement 
which is the first of its kind at the LHC, the only
ever performed using p-p collisions at 7 \TeV corresponding to
an integrated luminosity of 4.9 fb$^{-1}$, and a search for 
a neutral Higgs boson decaying to tau pairs using p-p collisions
with a dataset corresponding to an integrated luminosity of 
24.6 fb$^{-1}$, with 4.9 fb$^{-1}$ at 7 \TeV and 19.7 fb$^{-1}$ at 8 \TeV. 

%The measurement of the $\Wbb$ production
%cross section in p-p collisions at 7\TeV has been presented. 
The W+$\bbbar$ events were selected in the W $\to \mu\nu$ decay mode with a 
muon of $\pt>25\GeV$ and $|\eta|<2.1$, and two b jets with $\pt>25\GeV$ and $|\eta|<2.4$. 
The data sample corresponds to an integrated luminosity of $5.0\fbinv$. 
To extract the total number of $\Wbb$ events a maximum likelihood fit
was performed using a fit 
The final number of $\Wbb$ events were extracted via a binned maximum likelihood fit.
To constrain the most prominent backgrounds and reduce the final
systematic uncertainty the fit is performed simultaneously on the $\pt$ of the leading jet ($\rm{J_1}$) 
in the signal region after all selection requirements have been applied,
and on the $m_{\rm{J_3 J_4}}$ distribution obtained from the $\ttbar$ control region.
The  $\rm{J_1}$ $\pt$  is chosen as the final fit variable due to its discrimination power against top-related backgrounds. The measured cross section 
$\sigma(pp\rightarrow \mathrm{W} + \bbbar, p_T^{\mathrm{b}}>25~\GeV, |\eta^{\mathrm{b}}|<2.4)\times {\cal{B}}(\Wmn, p_T^{\mu}>25~\GeV, |\eta^{\mu}|<2.1) =0.53\pm  0.05\stat \pm 0.09 \syst \pm 0.06 \theo \pm 0.01\lumi pb.$
for production of a W boson in association with two b jets is in agreement with
the SM predictions.
This result is approaching the precision of theoretical predictions at NNLO, 
allowing a sensitive test of perturbative calculations
in the SM. It compliments previous W+b and W+bb results which
have shown varying levels of agreement 
with standard model predictions ~\cite{Aaltonen:2009qi,D0:2012qt,Aad:2011kp}. 
This measurement of $\Wbb$ also serves as an important
benchmark at the LHC to 
searches which include a single isolated lepton and one or more b jets in the final state,
(for example, the search for a neutral higgs boson also presented in this thesis) 
as $\Wbb$ is an irreducible background. %%run on sentence

%The search for neutral Higgs bosons decaying to tau pairs has been performed using events recorded by the CMS experiment at the LHC
%in 2011 and 2012 at a center-of-mass energy of 7 \TeV and 8 \TeV respectively. 
In the search for neutral Higgs bosons decaying to tau pairs, five different $\Pgt\Pgt$ final states are studied: 
$\Pe\Pgt_{h}, \Pgm\Pgt_{h}, \Pe\Pgm$, $\Pgm\Pgm$ and $\Pgt_{h}\Pgt_{h}$. 
To enhance the sensitivity to neutral Higgs bosons 
from the minimal supersymmetric extension of the standard model (MSSM), 
events containing zero and events containing one b-tagged jet are analyzed in separate categories.
No excess is observed in the tau-pair invariant-mass spectrum. 
Exclusion limits in the MSSM parameter space have been obtained 
for the $m_h^{\rm max}$ scenario. This search extends previous 
results to larger values of $M_A$ and excluded values of tan$\beta$ 
as low as $4.2$ at $M_A=140$~$\GeV$. 
In addition, model independent upper limits on the Higgs boson production cross 
section times branching fraction for gluon-gluon fusion and b-associated 
production are given.

\section{Future Outlook}
In 2015 the CMS experiment will have completed the necessary upgrades and 
 maintenance during long shutdown 1 (LS1) and 
will begin taking data produced in p-p collisions at 13\TeV. The run 
in 2015 is planned to collide protons at a bunch crossing rate of 25 ns 
with a peak luminosity of 2$\times$10$^{34}$ cm$^{-2}$s$^{-1}$. 
 It is likely that CMS will collect over 100 fb$^{-1}$ in 2015. 
 Thus far, the standard model of particle physics has been
 shown to perform well in describing experimental observations at energies around
 the electroweak scale of $O(246 \GeV)$.
 The recent discovery of a standard model-like higgs boson brought with it
 even more confidence in the standard model. However, questions arise when this model is seen as 
 a part of a grand unified theory. What happens when probing
 regions between the electroweak and planck scales $O(1.22 \times10^{19} \GeV)$ where
 the standard model encounters renormalization issues? 
 Supersymmetry has been proposed as a reliable method to 
 cancel out anomalies that arise at these high energies. 
%%in the 2015 run 
 Therefore, exciting times lie ahead as the search for physics beyond the standard model 
 continues.  