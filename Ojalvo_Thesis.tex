%%%%%%%%%%%%%%%%%%%%%%%%%%%%%%%%%%%%%%%%%%%%%%%%%%%%%%%%%%%%%%%%%%%%%%%%%%%%%
%%%% Preamble
%%%%%%%%%%%%%%%%%%%%%%%%%%%%%%%%%%%%%%%%%%%%%%%%%%%%%%%%%%%%%%%%%%%%%%%%%%%%%

%%%% The uwthesis.sty file relies on the memoir class!
%%%% You should be using the memoir class anyway; it makes life easier:
%%%% http://www.ctan.org/tex-archive/macros/latex/contrib/memoir/
\def\bbbar{\ensuremath{\mathrm{b\bar{b}}}}

\documentclass[oneside, letterpaper, 12pt, oldfontcommands]{memoir}

%%%% Import uwthesis.sty to get official formatting, then set your variables.
\usepackage{uwthesis}

\settitle{Bosons and b-Quark Jets}
\setauthor{Isobel Rose Ojalvo}
\setdepartment{Physics}
\doctors % or \masters
\setgraddate{2013}
\setdefensedate{1 February 2013} % or whatever format you want

%%%% Members of the Final Oral Committee (FOC)
%%%% Give name, rank, and department
%%%% 
\setfoca{Wesley Smith}{Professor}{Physics} % <- Your advisor
\setfocb{Sridhara Dasu}{Professor}{Physics}

%%%% Your abstract, used for the UMI abstract and in your front matter
\setabstract{%
  An awesome study of important things is presented.  
  I further describe my project here, but I will not exceed 350 words, 
  for that is strictly forbidden for the abstract of this document.  
  If you're submitting online, you can even put symbols 
  in your abstract and title, but you'll have to find out the HTML 
  character codes for the various symbols.
}

%%%%%%%%%%%%%%%%%%%%%%%%%%%%%%%%%%%%%%%%%%%%%%%%%%%%%%%%%%%%%%%%%%%%%%%%%%%%%
%%%% Document
%%%%%%%%%%%%%%%%%%%%%%%%%%%%%%%%%%%%%%%%%%%%%%%%%%%%%%%%%%%%%%%%%%%%%%%%%%%%%

\begin{document}

% Tell the memoir class to set up lowercase roman for pagination, etc.
\frontmatter

%%%% Uncomment this to create a UMI abstract page.
%%%% If you are submitting electronically, however, this page is unnecessary.
% \theumiabstract

% The title page
\thetitlepage
\clearpage

% The copyright page, if you want to pay the fee and register copyright.
\thecopyrightpage
\cleardoublepage

% These above pages should not be counted, so we reset the counter to 1.
\setcounter{page}{1}

% An abstract may be required by your department.
\section{Abstract}
\uwabstract
\cleardoublepage

% Acknowledgements go here if you want to include them.
\section{Acknowledgements}
This is where any acknowledgements would go.
\clearpage

% Table of contents
\maxtocdepth{subsection}
\tableofcontents* % the * means that there isn't an entry for the TOC itself
% \clearpage
% \listoffigures  % if you have any figures
% \clearpage
% \listoftables   % if you have any tables

% Tell the memoir class to set up normal pagination, etc. for the main doc
\mainmatter

\chapter{The Standard Model}
\section{A Historical Approach to High Energy Physics}
\section{Quarks, Leptons and Gauge Bosons}
\section{The Standard Model Higgs Boson}
\section{\bbbar Production at the LHC}

\chapter{Supersymmetry and MSSM}
\section{Minimally Supersymmetric Standard Model}
\section{MSSM Higgs Production}

\chapter{Experimental Setup}
\section{The Large Hadron Collider}
\section{Layout and Operation}
\section{Operating Conditions 2011 and 2012 Runs}

\chapter{The Compact Muon Solenoid Experiment}
\section{Coordinate System}
\section{Magnet}
\section{Tracking System}
The tracker is CMS's inner most detector. The purpose of the tracker is to reconstruct the
trajectories of charged particles coming from the LHC collisions and measure the charged particle momenta.
These charged particles leave a path hitherto refered to as a 'track'.%fix
These tracks are then used in the reconstruction of electrons, muons, taus, hadrons and jets 
and are also used to determine the primary vertex of an interaction. Additionally,
the tracker can be used in the identification of a 'secondary vertex' with in a jet,
a characteristic %fix
that is probable%fix
of a heavy (b or c -flavored) jet.
The CMS tracker consists of two main detectors: an inner silicon pixel detector and an outter
silicon strip detector.%%reference tracker figure 
Efficient reconstruction of collisions require a low hit occupancy, a high hit redundancy and a
fast response such that tracks can be identified reliably and attributed to the correct bunch crossing.
Low hit occupancy can be achieved with high granularity while high hit redundnancy requires
many detector layers. These former two requirements are only achieved with a 
high power density of on detector electronics which require efficient cooling. %fix?
Unfortunately, this is directly conflicting with the goal to limit the material 
budget of the tracker; interaction with material results in Coulomb scattering,
bremsstrahlung, nuclear interactions and photon conversion. Finally, an extremely
high particle flux results in radiation damage to the silicon sensors mainly 
in the form of modifications to the silicon crystal lattice. %% fix this to sound more positive
The aforementioned objectives and constraints resulted in a tracker design 
based entirely on silicon detector technology. 
At 5.4 m in length and 2.2 m in diameter, the CMS tracker is the largest inner
detector ever built in a high energy physics experiment.

The material budget of the tracker can be characterized by the rate at which a particle
passing through it loses energy. The radiation length, $X_{0}$, is the 
mean distance over which a high-energy electron loses all but 1/$e$ of its energy by bremssttrahlung %citation needed p.291 PDG ------
and $\frac{7}{9}$ of the mean free path for pair production by a high-energy photon. Figure %%reference figure with material budget ------
shows the material budget of the CMS tracker in terms of radiation length as a function of $\eta$. 
Due to the location of cabling, electronics and other services, the material budget of the 
tracker is at a minimum of 0.4 $X_{0}$ at $\eta \approx 0$ and increases to approximately 1.8 $X_{0}$
at $|\eta| \approx 1.4$, after which it decreases. 

%Resolution on the order of 10 micro meters http://arxiv.org/pdf/1007.1988.pdf
\subsection{Pixel Detector}%%%check to be sure where you say pixel system vs. pixel detector 
%%%%%Add a figure----------
The pixel detector is the inner most detector of the tracking system and, %with respect to the center interaction point ??? need?
covering the region from 4 to 15 cm in radius, is the closest detector to 
the interaction point. It has a high granularity and contributes precise 
tracking points in $r-\phi$ and $z$ and is therefore responsible for a small impact 
parameter resolution that is important for b and c-jet secondary
vertex reconstruction and $\tau$-lepton secondary vertex reconstruction.

The pixel detector is composed of individual pixel cells with a size of 
100 $\times$ 150 $\mu m^{2}$; it has 66 million active elements and covers
a surface area of 1 $m^{2}$. The pixel detector is composed of three barrel 
layers and two endcap disks which the pseudorapidity range from $-2.5<\eta<2.5$.
The three barrel layers are located at mean radii of 4.3, 7.3, and 10.2 cm. 
The endcap disks extend from 4.8 to 14.4 cm in radii are located at a mean distance
of $z=\pm35.5$cm and $z=\pm48.5$cm from the interaction point. 
As can be seen in figure XX this arrangement allows for 3 tracking points over %%add figure
almost the full $\eta$-range of the pixel system. Due to the geometric effect 
of particles entering the detector at an average angle of $20\circ$%check this
charge-sharing between pixels is achieved which improves position resolution.

The pixel system has a zero-suppressed read out scheme with analog pulse
height read-out. This improves the position resolution due to charge sharing,
it helps to separate signal and noise hits as well as to identify large hit 
clusters from overlapping tracks.
A position resolution on the order of 10 $\mu m$ is achieved.

\subsection{Silicon Strip Tracker}
The silicon strip detector is located outside the inner pixel detector, 
it extends from 25 cm to 110 cm in radius and pseudorapidity up to $|\eta|<2.5$. 
This region has a particle flux on 
the order of 100 times less than what is seen by the inner most layers of 
the pixel detector. It is a complementary system to the inner pixel
detector and has a lower granularity. The silicon strip detector 
has 9.3 million active elements over a total surface area of 198$m^{2}$
and consists of 3 large subsystems. As can be seen in figure XXXX %reference figure-----
The Tracker Inner Barrel and Disks (TIB/TID) extend in radius to
55cm and are composed of four barrel layers with three disks at each.
%%%%Mention Pitch here?????
The Tracker Outter Barrel (TOB) consists of six barrel layers and extends to $\pm118$ cm
in z. Extending beyond this in z the Tracker EndCaps (TEC+ and TEX- where the plus and minus
indicate the direction in z) are located from $124cm<|z|<280cm$. It is composed of 
nine disks which are populated with up to seven rings of radial-strip silicon detectors.
The combined layouts of the pixel detector and silicon strip detector
result in 8 to 14 high precision measurements of track impact points for 
$|\eta|<2.4$.


\chapter{Event Reconstruction}
\section{Tracks and Vertices}
\section{Electron Reconstruction and ID}
\section{Muon Reconstruction and ID}
\section{Tau Reconstruction and ID}
\section{Jet Reconstruction}
\subsection{Identification of Jets Originating from b-Quarks}
\section{Missing Transverse Energy}

\chapter{Event Simulation}

\chapter{Measurement of W plus \bbbar Pair Production}
\section{Initial Event Selection}
\section{Irreducible Backgrounds}
\section{Systematic Uncertainty}
\section{Signal Extraction}
\section{Results}

\chapter{Search for MSSM Higgs Boson}
\section{Initial Event Selection}
\section{Background Modeling}
\section{Systematic Uncertainty}
\section{Exclusion}
\section{Results}


\chapter{Conclusions}
\section{Future Outlook}

\end{document}



The Standard Model
-Background
-Quarks, Leptons, Gauge Bosons
-Standard Model Higgs
-bbbar production at the LHC

Supersymmetry and MSSM
-The Minimal Supersymmetric Standard Model
-MSSM Higgs production

Experimental Setup
- The LHC
- Layout and Operation
- Operating Conditions 2011/2012 runs

The Compact Muon Solenoid Experiment
-Coordinate System
-Magnet
-Tracking System
  --Pixel Detector
  --Silicon Strip Tracker
-Electromagnetic Calorimeter
-Hadronic Calorimeter
-Muon System
  --Drift Tube Chambers
  --Cathode Strip Chambers
  --Resistive Plate Chambers
-Trigger
-- L1 Trigger
--Regional Calorimeter Trigger
--High Level Trigger

Event Reconstruction
-Tracks and Vertices
-Electron Reconstruction and ID
-Muon Reconstruction and ID
-Tau Reconstruction and ID
-Jet Reconstruction
  -- Identification of Jets originating from b quarks
-Missing Transverse Energy

Measurement of W boson plus bbbar pair Production
-Systematics
-Backgrounds
-Signal Extraction
-Result

Search for MSSM Higgs Boson
-Systematics
-Backgrounds
-Limit
-Result

Conclusions
