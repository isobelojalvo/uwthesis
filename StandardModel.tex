\section{Quarks, Leptons and Gauge Bosons}
Following the advances of the 20th century the fundamental
particles in the SM are composed of 3 generations of quarks and leptons
where each generation has two particles. 

These 6 quarks and 6 leptons
are then mediated by the photon, $\gamma$, the W$^{+/-}$, the gluon

The Standard Model of particle physics follows an %%%
symmetry. As can be seen in figure %%figure
\subsection{The Standard Model}%%%Change name?
The idea that interactions are dictated by symmetry is of central
importance to the formulation of theoretical particle physics. 
Currently, it is the general belief that all particle interactions are
described by local gauge theories and that 
physical quantities (charge, color, lepton number, ect.) are 
always conserved.
To start the description of SM theory we remember that in classical 
mechanics Lagrange's equations of motion are used to
describe the motion of a particle or system of particles. %%ref goldstein
If we consider the generalized coordinates of a system, $q_{i}$,
and it's derivative as a function of time, $\dot{q_{i}}$, the equation
of motion can be written as,
\begin{equation}
0=\frac{d}{dt}\left(\frac{\partial L}{\partial\dot{q_{i}}}\right) - \frac{\partial L}{\partial q_{i}}
\end{equation}
where $L$ is defined as the difference of the kinetic and the potential energy, $L\equiv T-V$.
This equation equation can be formally extended to act on a wave function with %%I don't like "act on"
continuously varying coordinates $\phi({\bf x},t)$
where {\bf x} is the spatial coordinates and t is time.
In this regime, the Lagrangian equation of motion becomes
\begin{equation}
0=\frac{\partial}{\partial x_{\mu}}\left( \frac{\La}{\partial\left(\partial\phi/\partial x_{\mu}\right)}\right)-\frac{\partial\La}{\partial\phi}
\end{equation}
In the subsequent sections, following the standard methodology, $\La$ is referred to as the Lagrangian
despite it actually being the Lagrangian density,
\begin{equation}
L=\int{\La d^{3}x}
\end{equation}
As was stated in section XXX,%ref StandardModel Section 1
To derive the Lagrangian equation of motion for a given interaction
one can easily start from the Feynman Rules %%REF Srednicki
For a more formal approach, we begin by stating Noether's
Theorem that 

\subsection{The Higgs Mechanism and Electroweak Symmetry Breaking}
First we illustrate the Higgs Mechanism of a global gauge symmetry, 
which the mechanism that generates mass for the gauge bosons. 
We start by writing the Lagrangian, 
\begin{equation}
\label{eq:LU1}
\La=(\partial_{\mu}\phi)*(\partial^{\mu}\phi) - \mu^{2}\phi*\phi - \lambda (\phi*\phi)^{2}
\end{equation}
which describes the complex scalar field, $\phi=(\phi_{1}+i\phi_{2})/\sqrt(2)$.
This Lagrangian is invariant under the phase transformation $\phi\rightarrow e^{i\alpha}\phi$
(which is equivalent to stating that the (\ref{eq:LU1}) possesses a $U(1)$ global gauge symmetry).
Here, we consider the case where $\lambda>0$ and $\mu^{2}<0$. To illustrate the importance of (\ref{eq:LU1})
it is rewritten in the form
\begin{equation}
\La \equiv T-V = \frac{1}{2}(\partial_{\mu}\phi_{1})^{2}+\frac{1}{2}(\partial_{\mu}\phi_{2})^{2}-\frac{1}{2}\mu^{2}(\phi_{1}^{2}+\phi_{2}^{2})-\frac{1}{4}\lambda(\phi_{1}^{2}+\phi_{2}^{2})^{2}.
\label{eq:LU2}
\end{equation}
In (\ref{eq:LU2}) the potential term is
\begin{equation}
V=\frac{1}{2}\mu^{2}(\phi_{1}^{2}+\phi_{2}^{2})+\frac{1}{4}\lambda(\phi_{1}^{2}+\phi_{2}^{2})^{2}.
\end{equation}
Evaluating $\partial V/ \partial \phi = 0$ it is seen that the local minima
of the potential $V(\phi)$ is a circle in the $\phi_{1},\phi_{2}$ plane of radius $v$ where,
$\phi_{1}^{2}+\phi_{2}^{2}=v^{2}$ with $v^{2}= - \mu^{2}/\lambda$

\subsection{\bbbar Production at the LHC}


\subsection{Jet Hadronization and b-quarks}%%??
%quarks only live for $10^{-15}$ seconds