\section{Quarks, Leptons and Gauge Bosons}
Following the advances of the 20th century it has been experimentally 
determined that the fundamental
particles in the SM consists of 3 generations of quarks and leptons
where each generation has two particles. 
\begin{figure}[hb]
  \centering
	\includegraphics[width=\textwidth]{images/SMParticles2.png}
  	\caption[SM Particles]
   	{Standard Model Particles}
	\label{fig:SMParticles}
\end{figure}
Starting with the third 
generation which consists of the heaviest quarks, the top (t) and bottom (b),
and the $\tau$ lepton and $\tau$ neutrino ($\nu_{\tau}$).
The second generation consists of the charm (c) and strange (s) quarks,
the charged muon ($\mu$) and neutral muon neutrino ($\nu_{\mu}$).
Finally, the first generation which is composed of the lightest 
quarks, the up (u) and down (d) quarks,
and the lightest charged lepton with the longest lifetime, the electron (e), and its neutral 
neutrino partner, the electron neutrino ($\nu_{e}$). 
The mass, charge and spin can be seen in figure \ref{fig:SMParticles}.
%%%properties, mass, spin, charge, 

Each fundamental force is associated with spin 1 mediator particles.
The weak interactions are mediated by the $W^{\pm}$ and $Z$;
electromagnetic interactions are mediated by the photon and the strong interactions
are mediated by the 8 colored gluons. The Higgs Mechanism (described in the next section) 
is responsible for giving mass to the $W^{\pm}$, $Z$ and the fermions. 
The SM of particle physics describes very successfully the electroweak and strong
interaction of elementary particles over a wide range of energies.

The discovery of a Standard Model Higgs-like boson was announced jointly 
by the CMS and ATLAS collaborations in July of 2012. %%%Cite
As of the writing of this thesis the Higgs Boson has been observed at CMS via
its decay to $ZZ^{*}$, $\gamma\gamma$ and $\tau\tau$. The mass of this
Higgs Boson, as measured by CMS, is 125.6 $\pm$ 0.4(stat.) $\pm$ 0.2(syst.).

The Standard Model of particle physics follows a $SU(3)\times SU(2)_{L}\times U(1)$ %under field theory??
symmetry. The $SU(2)_{L}\times U(1)$ describes electroweak interactions.
The $SU(3)$ group describes color and the interactions
between gluons and quarks. In the unbroken $SU(2)_{L}\times U(1)$ symmetry
gauge bosons and fermions are massless. However, the recent
discovery of particle that closely resembles the Standard Model Higgs
Boson provides a Mechanism by which the symmetry is broken
and the $W^{\pm}$ and Z bosons, the quarks and leptons acquire mass. 
In the following sections the Higgs Mechanism is described in greater detail.
The extension of the Higgs to the Minimally Supersymmetric Standard Model
is described in the following chapter and the search for it along with
a measurement of the $\Wbb$ cross-section is the main purpose of this
thesis.
\subsection{Background to Lagrangian Construction}%%%Change name?
The idea that interactions are dictated by symmetry is of central
importance to the formulation of theoretical particle physics. 
Currently, it is the general belief that all particle interactions are
described by local gauge theories and that 
physical quantities (charge, color, weakspin, hyper charge) are 
always conserved.
To start the description of SM theory we remember that in classical 
mechanics Lagrange's equations of motion are used to
describe the motion of a particle or system of particles. %%ref goldstein
If we consider the generalized coordinates of a system, $q_{i}$,
and it's derivative as a function of time, $\dot{q_{i}}$, the equation
of motion can be written as,
\begin{equation}
0=\frac{d}{dt}\left(\frac{\partial L}{\partial\dot{q_{i}}}\right) - \frac{\partial L}{\partial q_{i}}
\end{equation}
where $L$ is defined as the difference of the kinetic and the potential energy, $L\equiv T-V$.
This equation can be formally extended to act on a wave function with %%I don't like "act on"
continuously varying coordinates $\phi({\bf x},t)$
where {\bf x} is the spatial coordinates and t is time.
In this regime, the Lagrangian equation of motion becomes
\begin{equation}
0=\frac{\partial}{\partial x_{\mu}}\left( \frac{\La}{\partial\left(\partial\phi/\partial x_{\mu}\right)}\right)-\frac{\partial\La}{\partial\phi}
\end{equation}
In the subsequent sections, following the standard methodology, $\La$ is referred to as the Lagrangian
despite it actually being the Lagrangian density,
\begin{equation}
L=\int{\La d^{3}x}
\end{equation}
To derive the Lagrangian equation of motion for a given interaction
one can easily start from the Feynman Rules.
However, in the following section we take a more formal approach.

\subsection{The Higgs Mechanism and Electroweak Symmetry Breaking}
In this section we illustrate the Higgs Mechanism and partially derive the
Standard Model Lagrangian. For an even more detailed approach see .%%%Ref Srednicki
First we consider the Higgs Mechanism of a global gauge symmetry, 
which the mechanism that generates mass for the gauge bosons. 
We start by writing the Lagrangian, 
\begin{equation}
\La=(\partial_{\mu}\phi)*(\partial^{\mu}\phi) - \mu^{2}\phi*\phi - \lambda (\phi*\phi)^{2}
\label{eq:LU1}
\end{equation}
which describes the complex scalar field, $\phi=(\phi_{1}+i\phi_{2})/\sqrt{2}$.
This Lagrangian is invariant under the phase transformation $\phi\rightarrow e^{i\alpha}\phi$
(which is equivalent to stating that (\ref{eq:LU1}) possesses a $U(1)$ global gauge symmetry).
Here, we consider the case where $\lambda>0$ and $\mu^{2}<0$. To illustrate the importance of (\ref{eq:LU1})
this Lagrangian is rewritten in the form
\begin{equation}
\La \equiv T-V = \frac{1}{2}(\partial_{\mu}\phi_{1})^{2}+\frac{1}{2}(\partial_{\mu}\phi_{2})^{2}-\frac{1}{2}\mu^{2}(\phi_{1}^{2}+\phi_{2}^{2})-\frac{1}{4}\lambda(\phi_{1}^{2}+\phi_{2}^{2})^{2}.
\label{eq:LU2}
\end{equation}
In (\ref{eq:LU2}) the potential term is
\begin{equation}
V=\frac{1}{2}\mu^{2}(\phi_{1}^{2}+\phi_{2}^{2})+\frac{1}{4}\lambda(\phi_{1}^{2}+\phi_{2}^{2})^{2}.
\end{equation}
Evaluating $\partial V/ \partial \phi = 0$ it is seen that the local minima
of the potential $V(\phi)$ is a circle in the $\phi_{1},\phi_{2}$ plane of radius $v$ where,
$\phi_{1}^{2}+\phi_{2}^{2}=v^{2}$ with $v^{2}= - \mu^{2}/\lambda$. This 
is of great importance as $(\phi_{1},\phi_{2})=(0,0)$ is not the ground state, as can be seen in figure .%%%%add mexican hat figure
Particle physics uses perturbation theory to calculate fluctuations 
about the minimum energy so, without loss of generality, the field $\phi$ is translated to a minimum
energy position $\phi_{1}=v$, $\phi_{2}=0$. The Lagrangian (\ref{eq:LU2}) is
expanded about the vacuum in terms of the fields $\eta$, $\xi$ by substituting
\begin{equation}
\phi(x)=\sqrt{\frac{1}{2}}[v+\eta(x)+i\xi(x)]
\label{eq:VTerms}
\end{equation}
into (\ref{eq:LU2}) and obtaining,
\begin{equation}
\La'=\frac{1}{2}(\partial_{\mu}\xi)^{2}+\frac{1}{2}(\partial_{\mu}\eta)^2+\mu^{2}+const.+\vartheta(\eta^{3},\xi^{3},\eta^{4},\xi^{4}).
\label{eq:LU3}
\end{equation}
The third term of this new Lagrangian, \ref{eq:LU3}, is a mass term
for the $\eta$-field, $m_{\eta}=\sqrt{-2\mu^{2}}$. The $\xi$-field has
a kinetic term (the first term in \ref{eq:LU3}) but no mass term
for $\xi$. This is a consequence of the Goldstone theorem which states
that whenever a symmetry is spontaneously broken, massless scalars occur.

Now that we have evaluated spontaneous breaking of 
a global gauge symmetry we can extend the formalism to the local $U(1)$ gauge symmetry.
First, we make the Lagrangian (\ref{eq:LU1}) invariant under a $U(1)$ 
local gauge transformation: $\phi\rightarrow e^{i\alpha(x)}\phi$
This is done by choosing a modified derivative, $D_{\mu}$, which will
transform covariantly under phase transformations, namely, 
$D_{\mu}\rightarrow e^{i\alpha(x)}D_{\mu}\psi$. Thus $\partial_{\mu}$
is replaced by $D_{\mu} = \partial_{\mu}-ieA_{\mu}$.
The gauge invariant Lagrangian in this instance is thus,
\begin{equation}
\La=(\partial^{\mu}+ieA^{\mu})\phi*(\partial_{\mu}-ieA_{\mu})\phi - \mu^{2} \phi*\phi - \lambda(\phi*\phi)^{2}-\frac{1}{4}F_{\mu\nu}F^{\mu\nu}.
\end{equation}
To avoid off diagonal terms instead of making the transformation \ref{eq:VTerms} 
we substitute in
\begin{equation}
\phi\rightarrow\sqrt{\frac{1}{2}}(\nu+h(x))e^{i\Theta(x)/\nu}
\end{equation}
and%%%be more verbose
\begin{equation}
A_{\mu}\rightarrow A_{\mu}+\frac{1}{e\nu}\partial_{\mu}\Theta
\end{equation}
Then by translating this field to a true global minimum
and, again, considering the solution where $\mu^{2}<0$ this Lagrangian becomes 
\begin{equation}
\La'=\frac{1}{2}(\partial_{\mu}h)^{2}-\lambda \nu^{2}h^{2}+\frac{1}{2}e^{2}\nu^{2}A_{\mu}^{2}-\lambda\nu h^{3}-
\frac{1}{4}\lambda h^{4} + \frac{1}{2}e^{2}A_{\mu}^{2}h^{2}+\nu e^{2}A_{\mu}^{2}h-\frac{1}{4}F_{\mu\nu}F^{\mu\nu}
\end{equation}
This equation describes the interaction between a vector gauge boson,
$A_{\mu}$, and a massive scalar, h. 
It is of importance to not that when examining this equation 
we see that the second term relates the mass 
of the higgs as $m_{h}=\sqrt{2\lambda v^{2}}$; along with this
the vector gauge boson acquires the mass, XXX.%%check this
To summarize this example of spontaneous breaking of a $U(1)$ gauge symmetry
results in a massless Goldstone boson which has been converted to an
additional longitudinal polarization degree of freedom. This results
in a Lagrangian with a massive scalar particle, a massive field and interaction terms.
This procedure is known as the Higgs mechanism. 

Next we consider spontaneous breaking of a $SU(2)$ gauge symmetry;
We begin with \ref{eq:LU1} and define $\phi$ as a $SU(2)$ doublet 
of a complex scalar field, %%fix wording
\begin{equation}
\phi=\left(
    \begin{array}{c}
      \phi_{\alpha} \\
      \phi_{\beta}
    \end{array}
  \right) =
  \sqrt{\frac{1}{2}} 
  \left(
    \begin{array}{c}
      \phi_{1}+i\phi_{2} \\
      \phi_{3}+i\phi_{4}
    \end{array}
  \right) .
  \label{eq:phidoublet}
\end{equation}
%\subsection{\bbbar Production at the LHC}
We must replace $\partial_{\mu}$ by the covariant derivative,
\begin{equation}
D_{\mu}=\partial_{\mu}+ig\frac{\tau_{a}}{2}W_{\mu}^{a},
\end{equation}
The gauge invariant Lagrangian becomes,
\begin{equation}
\La=
\left(\partial_{\mu}\phi + ig \frac{1}{2} \tau \cdot W_{\mu}\phi  \right)^{\dagger}
\left(\partial^{\mu}\phi + ig \frac{1}{2} \tau \cdot W^{\mu}\phi  \right)
-\mu^{2}\phi^{\dagger}\phi
-\lambda (\phi^{\dagger}\phi)^{2}
-\frac{1}{4}W_{\mu\nu}W^{\mu\nu}.
\label{eq:SU2L}
\end{equation}
Three gauge fields $W_{\mu}^{a}$ with a = 1,2,3 are included.
The potential in this Lagrangian is $V(\phi)=\mu^{2}\phi^{\dagger}\phi+\lambda (\phi^{\dagger}\phi)^{2}$.
As before, we are interested in the global minimum of this potential
in the case where $\mu^{2}<0$ and $\lambda>0$, which is
$\phi^{\dagger}\phi=-\mu^{2}/2\lambda$. Again, we are interested in the case where $\mu^{2}<0$ and $\lambda>0$
and expand about a point in $\phi(x)$ that is a global minimum,
\begin{equation}
\phi_{0}=\sqrt{\frac{1}{2}}
\left(
\begin{array}{c}
      0 \\
      \nu
    \end{array}
\right)
\label{eq:min}
\end{equation}
Due to gauge invariance, on can simply substitute,
\begin{equation}
\phi(x)=\sqrt{\frac{1}{2}}
\left(
\begin{array}{c}
      0 \\
      \nu+h(x)
    \end{array}
\right)
\label{eq:min}
\end{equation}
And in fact, if we substitute ($\ref{eq:min}$) into the Lagrangian (\ref{eq:SU2L}).
and find that the three bosons acquire a mass of $M=\frac{1}{2}g\nu$.

Finally, we need to formulate the Higgs mechanism in an $SU(2)\times U(1)$ 
gauge invariant Lagrangian so that the $W^{\pm}$ and $Z$ become massive. 
\begin{equation} %%link this to 14.65
\left| \left( i\partial_{\mu}+gT_{a}W_{\mu}^{1}+g'B_{\mu}\frac{Y}{2}\right)\phi \right|^{2} - \mu^{2}\left| \phi \right| ^{2}
+\lambda\left| \phi \right| ^{4}
\label{eq:L2}
\end{equation}
%%%%delete
We wish for (\ref{eq:L2}) to remain gauge invariant and 
for this Lagrangian to break the symmetries of $SU(2)$ and $U(1)_{Y}$
but leave $U(1)_{em}$, with generator $Q=T^{3}+\frac{Y}{2}$, unbroken. 
We can then make the same choice of \ref{eq:phidoublet} and a vacuum 
expectation value of \ref{eq:min}.
%then
%we choose $\phi$ such that,
%\begin{equation}
%\phi = 
%\left(
%\begin{array}{c}
%      \phi^{+} \\
%      \phi^{0}
%    \end{array}
%\right)
%\mathrm{where,}
%\begin{array}{c}
%      \phi^{+}\equiv (\phi_{1}+i\phi_{2})/\sqrt{2} \\
%      \phi^{0}\equiv (\phi_{3}+i\phi_{4})/\sqrt{2}
%    \end{array}
%\end{equation}
%%%deleted
Expanding the kinetic term of (\ref{eq:L2}) we have,
\begin{equation}
\left| \left(-g\frac{\tau}{2}\cdot W_{\mu}-i\frac{g'}{2}B_{\mu}\right)\phi\right|^{2}
\end{equation}
Then evaluating at (\ref{eq:min}),
\begin{equation}
=\frac{1}{8}\left|
\begin{pmatrix}
		gW_{\mu}^{3}+g'B_{\mu} & g(W_{\mu}^{1}-iW_{\mu}^{2})\\
		g(W_{\mu}^{1}+iW_{\mu}^{2}) & -gW_{\mu}^{3}+g'B_{\mu}
\end{pmatrix}
\begin{pmatrix}
0\\
\nu
\end{pmatrix}
\right|^{2}
\end{equation}
\begin{equation}
=
\frac{1}{8}\nu^{2}g^{2}\left[\left(W_{\mu}^{1} \right)^{2} \left(W_{\mu}^{2} \right)^{2} \right]
+\frac{1}{8}\nu^{2}(g'B_{\mu}-gW_{\mu}^{3})(g'B^{\mu}-gW^{3\mu}) 
\end{equation}
\begin{equation}
=(\frac{1}{2}\nu g)W_{\mu}^{+}W^{-\mu}+\frac{1}{8}(W_{\mu}^{3},B_{\mu})
\begin{pmatrix}
g^{2} & -gg'\\
-gg'  & g'^{2}
\end{pmatrix}
\begin{pmatrix}
W^{3\mu}\\
B^{\mu}
\end{pmatrix}
\end{equation}

We then can obtain the masses of the vector bosons.


\subsection{Jet Hadronization and b-quarks}%%??
%quarks only live for $10^{-15}$ seconds