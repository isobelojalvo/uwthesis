\section{Quarks, Leptons and Gauge Bosons}
Following the advances of the 20th century the fundamental
particles in the SM are composed of 3 generations of quarks and leptons
where each generation has two particles. 

These 6 quarks and 6 leptons
are then mediated by the photon, $\gamma$, the W$^{+/-}$, the gluon

The Standard Model of particle physics follows a $SU(3)\times SU(2)_{L}\times U(1)$ %under field theory??
symmetry. The $SU(2)_{L}\times U(1)$ describes electroweak interactions.
The $SU(3)$ group describes color and the interactions
between gluons and quarks. In the unbroken $SU(2)_{L}\times U(1)$ symmetry
gauge bosons and fermions are massless. However, the recent
discovery of particle that closely resembles the Standard Model Higgs
Boson provides a Mechanism by which the symmetry is broken
and the $W^{\pm}$ and Z bosons, the quarks and leptons acquire mass. 
In the following sections the Higgs Mechanism is described in greater detail.
The extension of the Higgs to the Minimally Supersymmetric Standard Model
is described in the following chapter and the search for it along with
a measurement of the $\Wbb$ cross-section is the main purpose of this
thesis.
\subsection{The Standard Model}%%%Change name?
The idea that interactions are dictated by symmetry is of central
importance to the formulation of theoretical particle physics. 
Currently, it is the general belief that all particle interactions are
described by local gauge theories and that 
physical quantities (charge, color, weakspin, hyper charge) are 
always conserved.
To start the description of SM theory we remember that in classical 
mechanics Lagrange's equations of motion are used to
describe the motion of a particle or system of particles. %%ref goldstein
If we consider the generalized coordinates of a system, $q_{i}$,
and it's derivative as a function of time, $\dot{q_{i}}$, the equation
of motion can be written as,
\begin{equation}
0=\frac{d}{dt}\left(\frac{\partial L}{\partial\dot{q_{i}}}\right) - \frac{\partial L}{\partial q_{i}}
\end{equation}
where $L$ is defined as the difference of the kinetic and the potential energy, $L\equiv T-V$.
This equation can be formally extended to act on a wave function with %%I don't like "act on"
continuously varying coordinates $\phi({\bf x},t)$
where {\bf x} is the spatial coordinates and t is time.
In this regime, the Lagrangian equation of motion becomes
\begin{equation}
0=\frac{\partial}{\partial x_{\mu}}\left( \frac{\La}{\partial\left(\partial\phi/\partial x_{\mu}\right)}\right)-\frac{\partial\La}{\partial\phi}
\end{equation}
In the subsequent sections, following the standard methodology, $\La$ is referred to as the Lagrangian
despite it actually being the Lagrangian density,
\begin{equation}
L=\int{\La d^{3}x}
\end{equation}
To derive the Lagrangian equation of motion for a given interaction
one can easily start from the Feynman Rules.
However, in the following section we take a more formal approach.

\subsection{The Higgs Mechanism and Electroweak Symmetry Breaking}
In this section we illustrate the Higgs Mechanism and partially derive the
Standard Model Lagrangian. For an even more detailed approach see .%%%Ref Srednicki
First we consider the Higgs Mechanism of a global gauge symmetry, 
which the mechanism that generates mass for the gauge bosons. 
We start by writing the Lagrangian, 
\begin{equation}
\La=(\partial_{\mu}\phi)*(\partial^{\mu}\phi) - \mu^{2}\phi*\phi - \lambda (\phi*\phi)^{2}
\label{eq:LU1}
\end{equation}
which describes the complex scalar field, $\phi=(\phi_{1}+i\phi_{2})/\sqrt{2}$.
This Lagrangian is invariant under the phase transformation $\phi\rightarrow e^{i\alpha}\phi$
(which is equivalent to stating that (\ref{eq:LU1}) possesses a $U(1)$ global gauge symmetry).
Here, we consider the case where $\lambda>0$ and $\mu^{2}<0$. To illustrate the importance of (\ref{eq:LU1})
this Lagrangian is rewritten in the form
\begin{equation}
\La \equiv T-V = \frac{1}{2}(\partial_{\mu}\phi_{1})^{2}+\frac{1}{2}(\partial_{\mu}\phi_{2})^{2}-\frac{1}{2}\mu^{2}(\phi_{1}^{2}+\phi_{2}^{2})-\frac{1}{4}\lambda(\phi_{1}^{2}+\phi_{2}^{2})^{2}.
\label{eq:LU2}
\end{equation}
In (\ref{eq:LU2}) the potential term is
\begin{equation}
V=\frac{1}{2}\mu^{2}(\phi_{1}^{2}+\phi_{2}^{2})+\frac{1}{4}\lambda(\phi_{1}^{2}+\phi_{2}^{2})^{2}.
\end{equation}
Evaluating $\partial V/ \partial \phi = 0$ it is seen that the local minima
of the potential $V(\phi)$ is a circle in the $\phi_{1},\phi_{2}$ plane of radius $v$ where,
$\phi_{1}^{2}+\phi_{2}^{2}=v^{2}$ with $v^{2}= - \mu^{2}/\lambda$. This 
is of great importance as $(\phi_{1},\phi_{2})=(0,0)$ is not the ground state, as can be seen in figure .%%%%add mexican hat figure
Particle physics uses perturbation theory to calculate fluctuations 
about the minimum energy so, without loss of generality, the field $\phi$ is translated to a minimum
energy position $\phi_{1}=v$, $\phi_{2}=0$. The Lagrangian (\ref{eq:LU2}) is
expanded about the vacuum in terms of the fields $\eta$, $\xi$ by substituting
\begin{equation}
\phi(x)=\sqrt{\frac{1}{2}}[v+\eta(x)+i\xi(x)]
\label{eq:VTerms}
\end{equation}
into (\ref{eq:LU2}) and obtaining,
\begin{equation}
\La'=\frac{1}{2}(\partial_{\mu}\xi)^{2}+\frac{1}{2}(\partial_{\mu}\eta)^2+\mu^{2}+const.+\vartheta(\eta^{3},\xi^{3},\eta^{4},\xi^{4}).
\label{eq:LU3}
\end{equation}
The third term of this new Lagrangian, \ref{eq:LU3}, is a mass term
for the $\eta$-field, $m_{\eta}=\sqrt{-2\mu^{2}}$. The $\xi$-field has
a kinetic term (the first term in \ref{eq:LU3}) but no mass term
for $\xi$. This is a consequence of the Goldstone theorem which states
that whenever a symmetry is spontaneously broken, massless scalars occur.

Now that we have evaluated spontaneous breaking of 
a global gauge symmetry we can extend the formalism to the local $U(1)$ gauge symmetry.
First, we make the Lagrangian (\ref{eq:LU1}) invariant under a $U(1)$ 
local gauge transformation: $\phi\rightarrow e^{i\alpha(x)}\phi$
This is done by choosing a modified derivative, $D_{\mu}$, which will
transform covariantly under phase transformations, namely, 
$D_{\mu}\rightarrow e^{i\alpha(x)}D_{\mu}\psi$. Thus $\partial_{\mu}$
is replaced by $D_{\mu} = \partial_{\mu}-ieA_{\mu}$.
The gauge invariant Lagrangian in this instance is thus,
\begin{equation}
\La=(\partial^{\mu}+ieA^{\mu})\phi*(\partial_{\mu}-ieA_{\mu})\phi - \mu^{2} \phi*\phi - \lambda(\phi*\phi)^{2}-\frac{1}{4}F_{\mu\nu}F^{\mu\nu}.
\end{equation}
To avoid off diagonal terms instead of making the transformation \ref{eq:VTerms} 
we substitute in
\begin{equation}
\phi\rightarrow\sqrt{\frac{1}{2}}(\nu+h(x))e^{i\Theta(x)/\nu}
\end{equation}
and%%%be more verbose
\begin{equation}
A_{\mu}\rightarrow A_{\mu}+\frac{1}{e\nu}\partial_{\mu}\Theta
\end{equation}
Then by translating this field to a true global minimum
and, again, considering the solution where $\mu^{2}<0$ this Lagrangian becomes 
\begin{equation}
\La'=\frac{1}{2}(\partial_{\mu}h)^{2}-\lambda \nu^{2}h^{2}+\frac{1}{2}e^{2}\nu^{2}A_{\mu}^{2}-\lambda\nu h^{3}-
\frac{1}{4}\lambda h^{4} + \frac{1}{2}e^{2}A_{\mu}^{2}h^{2}+\nu e^{2}A_{\mu}^{2}h-\frac{1}{4}F_{\mu\nu}F^{\mu\nu}
\end{equation}
%For the purpose of perturbative calculations a local minimum 
%\subsection{\bbbar Production at the LHC}


\subsection{Jet Hadronization and b-quarks}%%??
%quarks only live for $10^{-15}$ seconds