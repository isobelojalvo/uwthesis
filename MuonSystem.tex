\section{Muon System}
Good muon detection and resolution is of central importance 
to the CMS detector due it's appearance in the final state of many important processes
(for example H$\rightarrow$ZZ$\rightarrow\mu\mu$ and H$\rightarrow\tau\tau$ 
with $\tau\rightarrow\mu\overbar{\nu_{\mu}}\nu_{\tau}$). 
The muon is a relatively easy particle to detect due to its long life time 
and heavy mass they are less affected by radiative losses and the
CMS detector is capable of reconstruction muon momentum and charge over the entire
kinematic range of the LHC. The muon system covers the region in pseudorapidity
$|\eta|<2.4$, consists of about 25,000 m$^{2}$ and is made up of 3
 different types of gaseous particle detectors: drift tube (DT) chambers, 
 cathode strip chambers (CSC) and resistive plate chambers (RPCs).
 Their selection and location are determined by the environmental
 changes with respect to pseudorapidity. %%%% check this
 
\subsection{Drift Tube System}
The barrel region has a low flux of neutrons (which are left over 
from hadronic decays), a uniform 3.8 T magnetic field and a low 
muon rate. Drift tube chambers with standard rectangular drift cells, 
which are more sensitive to  these environmental effects but are %%% Along with RPC'S?? CHECK THIS!!!
relatively cheaper and easier to produce, are used in this region. 
They cover the pseudorapidity $|\eta|<1.2$ and are stationed within 
the iron yoke. The barrel muon detector consists of 4 concentric
cylinders of drift tube chambers around the beam line. The 3 inner
cylinders have 60 DT chambers while the outer has 70 DT chambers,
these combine to a total of approximately 172,000 sensitive wires.

A diagram of a drift cell is shown in figure %%%Drift cell figure

