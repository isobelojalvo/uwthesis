\chapter{Supersymmetry and MSSM}

In recent years it has already been seen that the standard model 
performs well in describing experimental observations at energies around
the electroweak scale of $O(246 \GeV)$.
The recent discovery of a standard model-like higgs boson brings with it
more confidence of the model. Questions arise when this model is seen as 
a part of a grand unified theory, for example, what happens when we probe
regions between the electroweak and planck scales $O(1.22 \times10^{19} \GeV)$? 
The standard model encounters difficulties in renormalization, this is known
as the Hierarchy Problem.

To illustrate these divergences we consider the radiative corrections to the higgs mass
from a fermion loop.
As seen in the previous chapter, the potential of the standard model higgs field can be
written as, 
\begin{equation}
V= \mu^{2}|\phi|^{2}+\lambda|\phi|^{4}
\end{equation}
where $\phi$ is a complex scalar field.
A non-vanishing vacuum expectation value of $246 \GeV$ is required to derive the masses of the $W/Z$
and the mass of the higgs which is defined as, $m_{h}=\sqrt{2\lambda v^{2}}$.

%When considering, for example, higher order radiative corrections to the mass of the higgs. For example, 
A fermion loop correction, seen in figure \ref{fig:fermionLoop}, 
adds the term $-\lambda_{f}\phi \bar{f}f$ to the Lagrangian.
This manifests as a correction to the higgs mass, namely,
\begin{equation}
\Delta m_{H}^{2}=-\frac{|\lambda_{f}|^{2}}{8\pi^{2}}\Lambda_{UV}^{2}+...
\label{eq:SUS1}
\end{equation}
%The loops represent terms which are proportional to $m_{f}^{2}$.
Where $\Lambda_{UV}$ is the ultraviolet cutoff. %%define UV cutoff
Then what happens when higher energy scales are considered and $\Lambda_{UV}\rightarrow \inf$?
Since the standard model is a renormalizable theory one could simply
pick $\Lambda_{UV}$ so that it regulates the loop. However, this would imply
that $\Lambda_{UV}$ is the energy scale at which new physics should enter.
To make matters worse, there are quantum corrections from the radiative
effects of every particle that couples to the higgs. If each of these 
particles gain their mass through the higgs mechanism then the entire standard model 
mass spectrum is sensitive to $\Lambda_{UV}$!

\section{MSSM}
To solve the Hierarchy Problem suppose there exists a massive scalar particle S, 
of mass $m_{S}$, that couples
to the higgs with a term $-\lambda_{S}\phi^{2} S^{2}$ as seen is 
figure \ref{fig:scalarLoop}. This massive scalar would give a correction of,
\begin{equation}
\Delta m_{H}^{2}=\frac{\lambda_{s}}{16\pi^{2}}\left[\Lambda_{UV}^{2}-2m_{s}^{2}ln(\Lambda_{UV}/m_{S})+... \right].
\label{eq:SUS2}
\end{equation}
By examining (\ref{eq:SUS1}) and (\ref{eq:SUS2}) it is apparent that if each of the quarks
and leptons is accompanied by a complex scalar with $\lambda_{S}=|\lambda_{f}|^{2}$  %OR TWO??
then the $\Lambda_{UV}^{2}$ terms will cancel. This model of symmetry between bosons
and fermions is known as Supersymmetry.

\begin{figure}[hb]
  \centering
  \begin{subfigure}[trim = 0mm 0mm 0mm 0mm, clip, width=3cm]{.4\textwidth}
	\marginbox{0mm 0pt 0mm 0pt}{\includegraphics[width=\textwidth]{images/fermionLoop.png}}
                %\marginbox{-10mm 0pt 10mm 0pt}{}
                \caption{}
                \label{fig:fermionLoop}
\end{subfigure}
\begin{subfigure}[trim = 0mm 0mm 0mm 0mm, clip, width=3cm]{.4\textwidth}
	\marginbox{0mm 0pt 0mm 0pt}{\includegraphics[width=\textwidth]{images/SLoop.png}}
	\caption{}
                %\marginbox{10mm 0pt -10mm 0pt}{}
                \label{fig:scalarLoop}
                	
  \end{subfigure}
   \caption[]{One loop quantum corrections for a fermion (\ref{fig:scalarLoop}) and a scalar (\ref{fig:scalarLoop}). }
\end{figure}

To define supersymmetry one needs a set of generators which 
transforms a fermionic state into a bosonic state and vice versa. The simplest 
operator which can perform these operations is a 2 component Weyl spinor $Q$
such that,
\begin{equation}
Q|\mathrm{Boson}\rangle=|\mathrm{Fermion}\rangle \qquad Q|\mathrm{Fermion}\rangle=|\mathrm{Boson}\rangle
\end{equation}.
The generators $Q$ and $Q^{\dagger}$ must obey the anticommutation
and commutation algebra,
\begin{equation}
\{Q,Q^{\dagger}\}=P^{\mu}
\end{equation}
\begin{equation}
\{Q,Q\}=\{Q^{\dagger},Q^{\dagger}\}=0
\end{equation}
\begin{equation}
[P^{\mu},Q ]=[P^{\mu},Q^{\dagger}]=0
\end{equation}.
where $P^{\mu}$ is the generator of space-time translations.

\section{Higgs Sector of the MSSM}
In the most simple supersymmetric extension of the standard model,
a two higgs doublet 
with an $SU(2)_{L}$ symmetry,
\begin{equation}
\Phi_{1}=
\left(
    \begin{array}{c}
      \phi_{1}^{0*} \\
      -\phi_{1}^{-}
    \end{array}
  \right) 
 \qquad
\Phi_{2}=
\left(
    \begin{array}{c}
      \phi_{2}^{+} \\
      -\phi_{2}^{0}
    \end{array}
  \right) 
\end{equation}
which gives mass to the quarks and charged leptons. The extra doublet
is needed to cancel out the corresponding supersymmetric higgs fermion
contributions.
$H_{1}^{0}$ and $H_{2}^{0}$ acquire vacuum expectation values $v_{1}$ and $v_{2}$ where
\begin{equation}
v = \sqrt{2}(v_{1}^{2}+v_{2}^{2})^{\frac{1}{2}}.
\end{equation}
The ratios of $v_{1}$ and $v_{2}$ is written as,
\begin{equation}
\tan(\beta)=\frac{v_{2}}{v_{1}}
\end{equation}
%%% something about shifting the neutral scalar fields by their vevs and diagonalization 
%The following higgs boson states are attained
\begin{equation}
\left(
    \begin{array}{c}
      H^{0}_{1} \\
     H^{0}_{2}
    \end{array}
  \right) 
  =
  \sqrt{2}
  \begin{pmatrix}
  \cos{\alpha} & \sin{\alpha}\\
  -\sin{\alpha}& \cos{\alpha}\\
  \end{pmatrix}
\left(
      \begin{array}{c}
      Re{\phi}_{1}^{0*}-v_{1} \\
     Re{\phi}_{2}^{0}-v_{2}
    \end{array}
  \right) 
\end{equation}
\begin{equation}
H_{3}^{0}=\sqrt{2}(\sin{\beta}Im\phi_{1}^{0*}+\cos{\beta}Im\phi^{0}_{2})
\end{equation}
\begin{equation}
H^{-}=(H^{+})^{*}=-\phi_{1}^{-}\sin{\beta}+\phi_{2}^{-}\cos{\beta}
\end{equation}

In the supersymmetric model the standard model and supersymmetric particles are arranged
into supermultiplets. These supermultiplets must contain both the fermion
and the boson superpartner. In the simplest approach, a  spin $\frac{1}{2}$
weyl fermion (for example, $e$) must have %%%check this
a spin-0 superpartner $\tilde{e}$, likewise, a supermultiplet with a spin-1 vector boson would have
 a spin $\frac{1}{2}$ weyl fermion superpartner.
 If supersymmetry remained unbroken
then $m_{e}=m_{\tilde{e}}$, as sparticles like $\tilde{e}$ have not 
yet been discovered this implies that supersymmetry must be a broken
symmetry.

In the MSSM the number of independent parameters can be reduced to 
$\tan{\beta}$ and the mass of the pseudoscalar, $m_{A}$. The masses of the other
neutral bosons are given by,
\begin{equation}
m_{h}^{2}=\frac{1}{2}\left(m_{A}^{2}+M_{Z}^{2}+\left[(m_{A}^{2}+
M_{Z}^{2})^{2}-4M_{Z}^{2}m_{A}^{2}\cos^{2}{(2\beta)}\right]^{\frac{1}{2}}\right).
\end{equation}
While the masses of the two charged bosons are,
\begin{equation}
m_{H^{\pm}}^{2}=m_{A}^{2}+M_{W}^{2}.
\end{equation}
Traditionally, searches for MSSM Higgs bosons are expressed in terms of benchmark scenarios where 
the lowest-order parameters tan$\beta$ and $M_A$ are varied, while fixing the other parameters that 
enter through radiative corrections to certain benchmark values. 
In this study, the $m_{h}^{\rm max}$ scenario~\cite{MHMAX-Carena,MHMAX-Carena-2002} is used as it 
yields conservative expected limits in the tan$\beta$ and $M_A$ plane. \\
%In order to make reliable phenomelogical predictions loop corrections must be included
%which depend on the masses of the supersymmetric particles. Accordingly, specific
%benchmark scenarios must be defined; this thesis uses the $m_{h}^{max}$ scenario
%referenced here \cite{}:\\
%\begin{equation}
\linebreak[4]
\begin{center}
$
M_{SUSY} = 1 \mathrm{TeV},
\mu=-200 \mathrm{GeV},
m_{\tilde{g}}=0.8 M_{SUSY}, $
\linebreak[4]
$
M_{A} \leq 1000 \mathrm{GeV},
X_{t}=2M_{SUSY},
A_{b} = A_{t}
$
%\linebreak[4]
\end{center}
Results in this thesis are interpreted both in the context of the MSSM 
$m_{h}^{\rm max}$ scenario and also in a model independent way, 
in terms of upper limits on $\sigma\cdot$BR($A\slash H\slash h\rightarrow\Pgt\Pgt$) for 
gluon-fusion and b-associated neutral Higgs boson production.

\section{MSSM Higgs Production}
\begin{figure}[hb]
\centering
  \begin{subfigure}[b]{.4\textwidth}
	\marginbox{-5mm 10pt 5mm 0pt}{\includegraphics[width=\textwidth]{images/ggA.png} }
	\end{subfigure}	
   \begin{subfigure}[b]{.4\textwidth}
	\marginbox{5mm 0pt -5mm 0pt}{\includegraphics[width=\textwidth]{images/bbA.png}}
    \end{subfigure}	
  	\caption[]
   	{MSSM Higgs Production diagrams}
	\label{fig:MSSMdiagrams}
\end{figure}

The dominant neutral MSSM Higgs boson production mechanism is the gluon-fusion process, 
$\Pg\Pg \to h, H, A$, for small and moderate values of tan$\beta$. At large values of tan$\beta$ 
the b-associated production is the dominant contribution, due to the enhanced bottom Yukawa 
coupling. In the region of large tan$\beta$ the branching ratio to tau leptons is enhanced, 
making the search for neutral MSSM Higgs bosons in the di-$\Pgt$ final state of particular interest. 
