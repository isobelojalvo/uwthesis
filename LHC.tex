\chapter{Large Hadron Collider}%%what is it? where is it? why is it?
The Large Hadron Collider (LHC) is a two ring superconducting hadron 
accelerator and collider. It is located on the border of Switzerland
and France to the northwest of the metropolitan area of Geneva.
The LHC is installed in a 26.7 km long tunnel which was originally constructed
from 1984 to 1989 for the Large Electron Positron (LEP) experiment. 
While a hadron hadron collider does not have the same limitations
due to synchroton radiation as an electron positron collider does, %%fix wording
the financial benefits of building the LHC in an existing collider tunnel 
warranted this decision. 
%
%The LHC uses superconducting magnets that operate at 2 K
The LHC is designed to operate at a center of mass energy of 14 TeV.
The overall purpose of constructing a hadron collider with such a high
center of mass energy is to expose the physics beyond the standard model.  

%The LHC is the product of an international collaboration and funded by a large number of country member states.
%%try not to start every sentence with "THE LHC"
\section{Layout}
The LHC follows the LEP tunnel geometry,
a schematic layout of the LHC is shown in figure %%ref figure
The tunnel is 2.7 m in diameter and houses a twin-bore magnet 
which provides both rings in the same structure.
As can be seen in figure%%ref figure
, the LHC can be schematically divided%%find better word
into 8 octants. At the center of each octant is a straight section and between
each of the 8 straight section there are 8 arcs. Each straight section
is 528 m long and can serve as an experimental point, where a beam
crossing occurs or as a utility insertion point.
ATLAS and CMS are both high luminosity experiments;
the ATLAS experiment is located at Point 1 and, on the opposite side
of the ring, CMS is located at Point 5. 
%%some info about LHCb and ALICE
% the LHC-B experiment at Point 2, ALICE experiment is at Point 8. 
Located at points 3 and 7 are collimation systems for beam cleaning,
the beam dump is at point 6 and point 4 contains to RF systems.

Each of the arcs that stretch between the straight sections
are made of 23 arc cells. Each arc cell is 106.9 m in length
and is comprised of two half cells each of which are 53.45 m long.

\section{Performance Goals and Constraints}
\section{Operation}
%Magnets
%RF System

\section{Operating Conditions in 2011 and 2012}

