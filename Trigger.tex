\section{Trigger}
The LHC delivers proton-proton collisions at the beam crossing 
interval of 25 ns, this corresponds to a crossing frequency of 40 MHz.
Each interaction, or event, requires 0.5 to 1 megabytes of storage space.
%%%%%%%%%fix here
As it is responsible for all data acquisition, %%fix
the trigger system is the most important subsystem of CMS. 

The CMS trigger system is divided into two parts: the Level 1 trigger 
and the High Level Trigger (HLT). 
The Level 1 trigger uses coarsely segmented
data from only the calorimeters and the muon system to make initial
data selections while holding high resolution data in pipelined memories
in front-end electronics. The Level 1 Trigger hardware is implemented in customized
programmable memory look up tables (LUTs), FPGAs and asics.
The Level 1 trigger must reduce rates by at least a factor of 10$^{6}$, 
resulting in a final output of approximately 30kHz,
before high resolution data is passed to the HLT.
The HLT has access to the complete readout of the event and can use
complex algorithms to filter events. For particularly interesting events, 
the algorithms used in this process are often similar to what is done in 
offline analysis. 
  \subsection{Level 1 Trigger}
The purpose of the Level 1 trigger is to reduce rates from the input
crossing rate of 40MHz to a maximum output rate of 100kHz. 
The first step in this process is to create
calorimeter trigger primitives using a calorimeter trigger primitive generator (TPG).
For the purpose of providing compact, fast information the calorimeters
are divided into trigger towers; the TPG sums the transverse energy the
ECAL crystals or HCAL read-out towers to obtain a trigger tower E$_{T}$.
For $|\eta|<$2.1, trigger towers have a granularity
in $\eta \times \phi$ of 0.087$\times$0.087, the granularity decreases at higher
$\eta$ values. 
The TPG electronics are integrated with the calorimeter readout; the trigger primitives
are then passed to the Regional Calorimeter Trigger (RCT) using
high-speed serial links.
An electron or photon candidate is expected to be narrow in $\eta$
and broader in $\phi$ therefore, an additional bit is used to indicate whether 
an electron candidate  
After that, the RCT combines tower information to form electron and 
photon candidates. The RCT also sums ECAL and HCAL towers into broader
regions which are used in the GCT to form jets. These regions also have a 
$\tau$-veto bit which determines how compatible a region is with a
$\tau$ lepton. After the GCT has combined regions to form jets, the GCT
then sorts the electron/photon candidates and jets and passes them onto
the Global Trigger.

The Muon Trigger uses all three muon systems (DTs,RPCs,CSCs) for triggering.
The DT system has excellent timing resolution and is used to reconstruct tracks
and associate tracks to bunch crossings. At a regional level, the DT and CSC track finders
combine segments to build muon tracks and assign transverse momentum using
LUTs. All regional information from the three muon systems are then forwarded
to the Global Muon Trigger (GMT) where it is then combined to proved track
information with equal or better resolution than the regional systems. The tracks
are then ranked and forwarded to the Global Trigger (GT).
  \subsection{Regional Calorimeter Trigger}
  \subsection{High Level Trigger}
After events pass the Level 1 trigger they are sent to the High Level Trigger (HLT). The
HLT uses a filter farm of processes to reduce rates by a factor of 10$^{6}$ from 
100 kHz to a final output rate of less than 300 Hz. 