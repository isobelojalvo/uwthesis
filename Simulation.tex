\chapter{Event Simulation}
\section{Physics Event Generation}
\section{Hard Scattering Process}
%Say something about simulation of hard processes
We wish to use an event generator which produces hypothetical events with a distribution
that is predicted by theory.
%list issues
It is easiest to address these issues by considering the simulation of 
the process $W\rightarrow\mu\nu$. 
\begin{equation}
d\sigma(u\bar{d}\rightarrow W^{-}\rightarrow \mu^{-}\bar{\nu_{\mu}})
=\frac{1}{2\hat{s}}|\mathcal{M}(u\bar{d}\rightarrow W^{-}\rightarrow \mu^{-}\bar{\nu_{\mu}})|^{2}
\frac{d\cos\theta d\phi}{8(2\pi)^{2}}
\label{eq:matrixEle}
\end{equation}
where the decay angles $(\theta,\phi)$ of the $W^{-}$ are the two degrees of
freedom of the problem. $\mathcal{M}$ is the relevant matrix element and $\hat{s}$
is the center of mass energy squared.
%In typical event generators which simulate hard processes 
%ref equation
Equation (\ref{eq:matrixEle}) is used to write an event generator. 
To do this first a sampling of the relevant phase
space must be done using a random number generator, next the events must
be unweighted using the hit-and-miss technique to produce events with observables. %%this is terrible
The relevant phase space for this process is two dimensional: -1 < $cos\theta$ < 1,
0 < $\phi$<2$\pi$. The values of $cos\theta$ and $\phi$ can be chosen using
a uniform random number generator. 
Evaluating the equation (\ref{eq:matrixEle}) at $d\sigma(\theta_{i},\phi_{i})$ gives
that candidate event's differential cross section which is equivalent to the 
probability of this event occurring. The average of many candidate event's differential
cross section, $\langle d\sigma\rangle$, is an approximation to $\int d\sigma$ and converges
to the measured cross section.

So far the candidate events are distributed flat in phase space and there is 
no physics information in the distributions.
The next step then is to unweight the event using the 'hit-and-miss' technique
so that they are distributed according to theoretical prediction. By unweighting
the events are generated with the frequency predicted by the theory 
being modeled and individual events represent what might be observed in a
trial experiment. 

%Parton Distribution Functions
%Parton Showering and Hadronization
%Monte Carlo Generator Programs
%K-factors

%Detector Simulation
\subsection{Detector Simulation}
