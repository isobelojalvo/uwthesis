\chapter{Event Simulation}
%\section{Physics Event Generation}
\section{Hard Scattering Process}
%Say something about simulation of hard processes
We wish to use an event generator which produces hypothetical events with a distribution
that is predicted by theory.
%list issues
It is easiest to address these issues by considering the simulation of 
the process $W\rightarrow\mu\nu$. 
\begin{equation}
d\sigma(u\bar{d}\rightarrow W^{-}\rightarrow \mu^{-}\bar{\nu_{\mu}})
=\frac{1}{2\hat{s}}|\mathcal{M}(u\bar{d}\rightarrow W^{-}\rightarrow \mu^{-}\bar{\nu_{\mu}})|^{2}
\frac{d\cos\theta d\phi}{8(2\pi)^{2}}
\label{eq:matrixEle}
\end{equation}
where the decay angles $(\theta,\phi)$ of the $W^{-}$ are the two degrees of
freedom of the problem. $\mathcal{M}$ is the relevant matrix element and $\hat{s}$
is the center of mass energy squared.
%In typical event generators which simulate hard processes 
%ref equation
Equation (\ref{eq:matrixEle}) is used to write an event generator. 
To do this first a sampling of the relevant phase
space must be done using a random number generator, next the events must
be unweighted using the hit-and-miss technique to produce events with observables. %%this is terrible
The relevant phase space for this process is two dimensional: -1 < $cos\theta$ < 1,
0 < $\phi$<2$\pi$. The values of $cos\theta$ and $\phi$ can be chosen using
a uniform random number generator. 
Evaluating the equation (\ref{eq:matrixEle}) at $d\sigma(\theta_{i},\phi_{i})$ gives
that candidate event's differential cross section which is equivalent to the 
probability of this event occurring. The average of many candidate event's differential
cross section, $\langle d\sigma\rangle$, is an approximation to $\int d\sigma$ and converges
to the measured cross section.
So far the candidate events are distributed flat in phase space and there is 
no physics information in the distributions.
The next step then is to unweight the event using the 'hit-and-miss' technique
so that they are distributed according to theoretical prediction. By unweighting
the events are generated with the frequency predicted by the theory 
being modeled and individual events represent what might be observed in a
trial experiment. 

%Parton Distribution Functions
%Parton Showering and Hadronization
\subsection{Parton Showering, Hadronization and the Underlying Event}

Parton showering and hadronization is a form of correction to the hard scattering subprocess
where it is not feasible to calculate these processes exactly.

After the hard processes are generated
%1.)
partons undergo showering 
Parton showers provide an %%exclusive?
description of the event: at leading order the transverse momentum
for the $W$ will always be zero since there is nothing for the $W$
to recoil against.
The beam remnants are the colored remains of the proton which
are left behind when the parton which participates in the hard subprocess
is pulled out. As the beam remnants are color connected to the hard
subprocess they should be included in the hadronization system. 
Multiple parton interactions (MPI) where more than one pair
of beam partons interact are also simulated. In the $\Wbb$ analysis
MPI agreement between data and simulation is tested 
by observing the ratio of $(p_{T,j_{1}}+p_{T,j_{2}})/p_{T,j_{1},j_{2}}$
Finally, pileup from other proton-proton collisions are added to the event.
\section{Monte Carlo Generator Programs}

%K-factors

\subsection{Detector Simulation}
