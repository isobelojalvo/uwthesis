\chapter{Event Simulation}

%\section{Physics Event Generation}
\section{Hard Scattering Process}
%Say something about simulation of hard processes
In order to 
We wish to use an event generator which produces hypothetical events with a distribution
that is predicted by theory.
%list issues
It is easiest to address these issues by considering the simulation of 
the process $W\rightarrow\mu\nu$. 
\begin{equation}
d\sigma(\bar{u}d\rightarrow W^{-}\rightarrow \mu^{-}\bar{\nu_{\mu}})
=\frac{1}{2\hat{s}}|\mathcal{M}(u\bar{d}\rightarrow W^{-}\rightarrow \mu^{-}\bar{\nu_{\mu}})|^{2}
\frac{d\cos\theta d\phi}{8(2\pi)^{2}}
\label{eq:matrixEle}
\end{equation}
where the decay angles $(\theta,\phi)$ of the $W^{-}$ are the two degrees of
freedom. $\mathcal{M}$ is the relevant matrix element and $\hat{s}$
is the center of mass energy squared.
%In typical event generators which simulate hard processes 
%ref equation
Equation (\ref{eq:matrixEle}) is used to write an event generator. 
To do this first a sampling of the relevant phase
space must be done using a random number generator, next the events must
be unweighted using the hit-and-miss technique to produce events with observables. %%this is terrible
The relevant phase space for this process is two dimensional: -1 $< \cos\theta <$ 1,
0 $< \phi < $ 2$\pi$. The values of $cos\theta$ and $\phi$ can be chosen using
a uniform random number generator. 
Evaluating the equation (\ref{eq:matrixEle}) at $d\sigma(\theta_{i},\phi_{i})$ gives
that candidate event's differential cross section which is equivalent to the 
probability of this event occurring. The average of many candidate event's differential
cross section, $\langle d\sigma\rangle$, is an approximation to $\int d\sigma$ and converges
to the measured cross section.
So far the candidate events are distributed flat in phase space and there is 
no physics information in the distributions.
The next step then is to unweight the event using the 'hit-and-miss' technique
so that they are distributed according to theoretical prediction. By unweighting
the events are generated with the frequency predicted by the theory 
being modeled and individual events represent what might be observed in a
trial experiment. 

%Parton Distribution Functions
%Parton Showering and Hadronization
\subsection{Parton Showering, Hadronization and the Underlying Event}

Parton showering is a form of correction to the hard scattering subprocess
where it is not feasible to calculate these processes exactly.
Programs that use the parton shower approach are also
able to simulate a wide variety of initial and final state processes.
%%Lund String Model
In a proton-proton interaction a parton is resolved
at scale $Q$ and momentum fraction $x$ from each
proton. Figure %%insert figure of event evolution
shows a proton-proton collision where a valence quark
is separated from one of the colliding protons and a 
a sea anti-quark is separated from the other of the colliding protons.
The phenomenology of the parton interaction is encoded
in the parton distribution function $f(x,Q^{2})$. 
The parton distribution function represents the probability
densities to find a parton carrying
a momentum faction $x$ at a squared energy
scale $Q^{2}$. 
Since the hadronization scale is much smaller than the hard scale the impact
of the hadronization model choice on the final state is typically
small for most physics analyses. 
Parton showers are essential in providing a %%exclusive?
description of the event: at leading order the transverse momentum
for the $W$ will always be zero since there is nothing for the $W$
to recoil against.

The beam remnants are the colored remains of the proton which
are left behind when the parton which participates in the hard subprocess
is pulled out. As the beam remnants are color connected to the hard
subprocess they should be included in the hadronization system. 
Multiple parton interactions (MPI) where more than one pair
of beam partons interact are also simulated. In the $\Wbb$ analysis
MPI agreement between data and simulation is tested 
by observing the ratio of $(p_{T,j_{1}}+p_{T,j_{2}})/p_{T,j_{1},j_{2}}$
Finally, pileup from other proton-proton collisions are added to the event.
%b quarks are not expected to be
%produced at any signi�cant rate in nonperturbative processes [1], and they do not occur
%as valence �avours of the commonly used beam particles. A priori, they are therefore
%excellent probes of the underlying hard dynamics, whether that involves standard QCD
%processes or various kinds of new physics
\section{Monte Carlo Generator Programs}
A number of monte carlo (MC) generators are used to simulate
physics processes. 
The choice of MC generator is largely dependent on the 
type of process to be simulated. %fix
%K-factors
\subsection{MadGraph}
MadGraph is is a matrix element generator for SM processes at
any collider. It provides a computation of tree-level matrix elements
with a fixed number of partons in the final state. 
For a user-specified process MadGraph generates
the amplitudes for all relevant subprocesses and produces
the mappings for the integration over the phase space. 
The final state only 
Events may be passed directly to a shower MC program.
\subsection{Tauloa}
Tauloa is a package which is used for generation of tau lepton decays including
spin polarization. For each tau decay mode there is a individual phase space generator,
modeling of the weak decay including first order QED corrections for leptonic decays
and a part describing the hadronic current.
\subsection{PYTHIA}
At leading order \PYTHIA contains approximately 240 different $2\rightarrow n$ subprocesses.
The initial state shower is based on backwards evolution whereby the 
hard scattering process is simulated and is evolved backwards in time to the shower
initiators. Partons radiated in the initial state can initiate final-state showers
of their own. The Lund string model is used in hadronization; it is based on linear confinement
where quarks  are located at the ends of the string and gluons
are energy and momentum carrying kinds in the string. The production of a new $q\bar{q}$
pair causes the string to break. A quark from one break can combine with an antiquark
from an adjacent quark to form a meson. 
\section{Detector Simulation}
\GEANTfour is a toolkit for simulating the passage of particles through matter.
In particular it provides tracking of particles taking account of the geometry, fields and physics processes.
\GEANTfour uses decay tables for particles such as $\pi$, $K$ mesons
and resonant baryons based on data from the Particle Data Group.
