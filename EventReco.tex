\chapter{Event Reconstruction and Monte Carlo Simulations}
After data is collected it undergoes 'Event Reconstruction'.
At this stage %particles
%a series of algorithm to identify and particles and event variables

\section{Track and Primary Vertex Reconstruction}
%%%%%%%%%%%%%%%%Track Reconstruction [31] %TRK-11-001
In a technique known as Combined Track Finding (CTF),
the collection of reconstructed tracks is produced via
multiple iterations of the CTF track reconstruction sequence.
The first three iterations are meant to reconstruct successively
lower $p_{T}$ and quality tracks.%%not quite right actually...
while iterations four through six recover any tracks 
not found by previous iterations.

The CTF track reconstruction proceeds as follows:
%seed generation
%many electrons lose a significant fraction of their energy to bremsstrahlung radiation in the tracker
%Therefore, to ensure high efficiency, track finding begins with trajectory seeds created in the inner region of the tracker
%Seed generation requires information on the reconstructed beam spot position
%The tracks and primary vertices found with this algorithm are known as pixel tracks and pixel vertices, respectively.
\item First, seeds are generated which provides the initial track candidates.
Charged particles follow helical paths in the magnetic field therefore five parameters (including curvature)
are needed to define a trajectory; to determine these five parameters at least
3 hits or 2 hits in the pixel detector and a constraint on the origin of the track trajectory from the beam
spot is required. Seeds are built in the inner part of the tracker
and track candidates are reconstructed outwards. This is due to a higher finer granularity 
and hence lower occupancy in the center of the pixel detector. Each iteration of CTF
uses independent quality parameters for seeding layers. %%input table?
%
\item Next, track finding is performed based on the Kalman filter method. %%need reference TRK-11-011 [42]
Track finding starts by using the seed trajectories to define and then search
for adjacent layers of the detector with a hit. The next step provides the possibility
of adding an invalid hit in the case where the particle failed to produce a hit.
Finally the track finding algorithm updates the trajectories of the tracks. 
%track fitting by means of kalman filter and smoother
\item Track fitting then is used to adjust for the possibility of bias added during the track
finding stage. The trajectory is therefore refitted using a Kalman filter and smoother with
a Runge-Kutta propagator that takes into account both material effect and accomodates
for a inhomogeneous magnetic field.
%track selection sets quality flags and discards tracks that fail certain criteria
\item

%Kalman filter [32] 

%Iterative tracking?? maybe move to b-Jet ID
%Vertex reconstruction [33], [34] 
%Deterministic Annealing
\section{Electron ID and Reconstruction}
%Ecal Seeding or track seeding??

%GSF[38]

% Rejection of Electrons from converted photons [39]

%electron ID ---twiki table

%Electron trigger?? maybe
\section{Muon ID and Reconstruction}
%Muon reco [40]

%standalone, global, tracker

%Muon ID (use old muon plots?) mention muon veto in analysis

%Muon trigger

%\section on PF??
% section on lepton isolation?

\section{\tau ID and Reconstruction}
%Tau ID 
%Tau Isolation

\section{Jet ID and Reconstruction}
%Particle flow
%Anti-Kt algo [43]

%Jet energy corrections [44]
\subsection{b-Jet ID}
%CSV algorithm
%SV in jets
\section{Missing Transverse Energy}
%met 
%any documentation