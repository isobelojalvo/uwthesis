\chapter{Event Reconstruction and Monte Carlo Simulations}
After data is collected it undergoes 'Event Reconstruction'.
At this stage %particles
%a series of algorithm to identify and particles and event variables

\section{Track and Primary Vertex Reconstruction}
%%%%%%%%%%%%%%%%Track Reconstruction [31] %TRK-11-001

%write an intro
%Track selection involves selection of tracks which are consistent with being produced promptly in the primary interaction region by imposing requirements on the maximum value of significance of the transverse impact parameter relative to the beam spot
\subsection{Track Reconstrucion}
In a technique known as Combined Track Finding (CTF),
the collection of reconstructed tracks is produced via
multiple iterations of the CTF track reconstruction sequence.
The first three iterations are meant to reconstruct successively
lower $p_{T}$ and quality tracks.%%not quite right actually...
while iterations four through six recover any tracks 
not found by previous iterations.

After each iteration the CTF track reconstruction sequence, the hits
associated with tracks are removed; this reduces the combinatorial complexity
and simplifies subsequent iterations in a search for more difficult classes of track (for example,
low $p_{t}$ or which are greatly displaced).
The CTF track reconstruction proceeds as follows:
\begin{itemize}
\item First, seeds are generated which provides the initial track candidates.
Charged particles follow helical paths in the magnetic field therefore five parameters (including curvature)
are needed to define a trajectory; to determine these five parameters at least
3 hits or 2 hits in the pixel detector and a constraint on the origin of the track trajectory from the beam
spot is required. Seeds are built in the inner part of the tracker
and track candidates are reconstructed outwards. This is due to a higher finer granularity 
and hence lower occupancy in the center of the pixel detector. Each iteration of CTF
uses independent quality parameters for seeding layers. %%input table?
%
\item Next, track finding is performed based on the Kalman filter method. %%need reference TRK-11-011 [42]
Track finding starts by using the seed trajectories to define and then search
for adjacent layers of the detector with a hit. The next step provides the possibility
of adding an invalid hit in the case where the particle failed to produce a hit.
Finally the track finding algorithm updates the trajectories of the tracks. 
%track fitting by means of kalman filter and smoother
\item Track fitting then is used to adjust for the possibility of bias added during the track
finding stage. The trajectory is therefore refitted using a Kalman filter and smoother with
a Runge-Kutta propagator that takes into account both material effect and accomodates
for a inhomogeneous magnetic field.
%track selection sets quality flags and discards tracks that fail certain criteria
\item The final step is track selection where tracks are required to pass a number of quality
based selection criteria. 
The selection criteria puts a requirement on a track's number of layers %fix
with valid hits, the fit-based $\chi^{2}/dof$, and the track's compatibility
with a primary vertex (PV). In addition to these, several requirements are
imposed as a function of track $p_{T}$, $\eta$, and the number of layers
with valid hits. 
\end{itemize}
The tracks and primary vertices found with this algorithm are known as pixel 
tracks and pixel vertices, respectively. The performance of the track reconstruction
offline software is shown here %%add tracking plots
%Kalman filter [32] 
\subsection{Primary Vertex Reconstrucion}
Primary vertices (PVs) are reconstruct in order to locate and determine the associated uncertainty of all
proton-proton interaction vertices regardless of whether it is a 'signal' or 'background' vertex.
Primary vertex reconstruction proceeds in three steps.

The first step is to select tracks based on their association with a primary interaction region.
To do this, a number of quality selections are imposed: they are based upon the significance of the
transverse impact parameter ($d_{xy}$), the number of strip and pixel hits that are 
associated with a track and the normalized $\chi^{2}$ from the fit to the
trajectory. In selecting tracks their is no requirement on the $p_{T}$ of the track; 
this is important so that all PVs including ones from minimum bias events will be reconstructed.

The second step is to cluster the selected tracks based on their $z$ coodinate at their
point of closest approach to the beam spot. This is done using a Deterministic Annealing (DA)
algorithm which finds a global minimum given many degrees of freedom.%%ref DA, More on DA?

The third and final step is to take candidate vertices based on DA clustering in $z$ and use
an 'adaptive vertex fitter' %ref
to compute vertex parameters. These parameters include the 3-D position and covariance matrix,
as well as indicators for the success of the fit such as the number of degrees ($n_{dof}$) of freedom for
each vertex and weights of the track used in each vertex. The adaptive vertex fitter
uses a modified definition of $n_{dof}$ where,
\begin{equation}
n_{dof} = -3+2\sum^{nTracks}_{i=1} w_{i},
\end{equation}
Here, $w_{i}$ is the weight of the $i^{th}$ track. This implies that $n_{dof}$ is strongly
correlated with the number of tracks that are compatible with arising from the interaction
region which means that $n_{dof}$ can be also be used to select true proton-proton interactions.
%Vertex reconstruction [33], [34] 
\section{Electron ID and Reconstruction}
%Ecal Seeding or track seeding??

%GSF[38]

% Rejection of Electrons from converted photons [39]

%electron ID ---twiki table

%Electron trigger?? maybe
\section{Muon ID and Reconstruction}
%Muon reco [40]

%standalone, global, tracker

%Muon ID (use old muon plots?) mention muon veto in analysis

%Muon trigger

%\section on PF??
% section on lepton isolation?

\section{$\tau$ ID and Reconstruction}
%Tau ID 
%Tau Isolation

\section{Jet ID and Reconstruction}
%Particle flow
%Anti-Kt algo [43]

%Jet energy corrections [44]
\subsection{b-Jet ID}
%CSV algorithm
%SV in jets
\section{Missing Transverse Energy}
%met 
%any documentation